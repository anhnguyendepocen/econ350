%Input premable
\documentclass{beamer}

%Packages
\usepackage{graphicx}
\usepackage{graphics}
\usepackage{hyperref}
\usepackage[english]{babel}
\usepackage{amsmath}
\usepackage{amsfonts}
\usepackage{amssymb}
\usepackage{bm}
\usepackage{comment}
\usepackage{etoolbox}
\usepackage{graphicx}
\usepackage{tabularx,ragged2e,booktabs}
\usepackage{caption}
\usepackage{hyphenat}
\usepackage{fixltx2e}
\usepackage[para]{threeparttable}
\usepackage[capposition=top]{floatrow}
\usepackage{subcaption}
\usepackage{pdfpages}
\usepackage{natbib}
\usepackage{rotating}

%Math Functions
\DeclareMathOperator{\var}{Var}
\DeclareMathOperator{\cov}{Cov}

%Commands
\newcommand\independent{\protect\mathpalette{\protect\independenT}{\perp}}
\def\independenT#1#2{\mathrel{\rlap{$#1#2$}\mkern2mu{#1#2}}}
\newcommand{\overbar}[1]{\mkern 1.5mu\overline{\mkern-1.5mu#1\mkern-1.5mu}\mkern 1.5mu}
\newcommand{\equald}{\ensuremath{\overset{d}{=}}}

\newenvironment{wideitemize}{\itemize\addtolength{\itemsep}{10pt}}{\enditemize}

\mode<presentation>
	{
	\usetheme{UChicagoJorge}
	\setbeamercovered{transparent = 28}
	}

%\usecolortheme{UChicagoJorge} 
%\useinnertheme{UChicagoJorge}
%\useoutertheme{UChicagoJorge}

\title{Prices, Market Wages, and Labor Supply}
\subtitle{James J. Heckman (1974), Econometrica}
\author{Econ 350\institute{The University of Chicago, Economics}}

\begin{document}


\begin{frame}[plain]
	\titlepage
\end{frame}


\AtBeginSection[]
{
   \begin{frame}
       \frametitle{Outline}
       \tableofcontents[currentsection]
   \end{frame}
}

\section{About the paper}

\begin{frame}
	\frametitle{About the paper}
		\begin{itemize}
 			\item Pretty cool paper
 			\item Why?
 				\begin{itemize}
 					\item Models wage rates, hours worked and the decision to work all together! 
 					\item First time someone did this in the field!
 					\item Specially important for women's labor supply
 				\end{itemize}
 			\item How?
 				\begin{itemize}
 					\item Focus on women's labor supply
 					\item Derives a common set of parameters which underlie the functions determining 
 					\begin{enumerate}
 						\item the probability that a woman works
 						\item her hours of work
 						\item her observed wage rate
 						\item her asking wage rate or shadow price of time
 					\end{enumerate}
 				\end{itemize}
 		\end{itemize}
\end{frame}

\section{Model}

\begin{frame}
	\frametitle{Wages}
		\begin{itemize}
			\item Shadow wage (value of marginal units of wife's time in household production and consumption) function depends on
				\begin{itemize}
					\item hours of work (time in no-market activities, $h$
					\item wage of husband, $W_{m}$
					\item vector of goods prices, $P$
					\item asset income of the household, $A$
					\item constraints from previous economic decision, $Z$ (e.g., number of children)
					\begin{equation}
						W^* = g \left( h, W_{m}, P, A, Z \right)
					\end{equation}
				\end{itemize}
			\item Market wage depends on
				\begin{itemize}
					\item Education, $E$
					\item Experience, $S$
						\begin{equation}
							W = B (E,S)
						\end{equation}
				\end{itemize}
		\end{itemize}
\end{frame}

\begin{frame}
	\frametitle{Decision}
		\begin{itemize}
			\item Assume that the woman is free to adjust her working hours
			\begin{itemize}
				\item If she works
					\begin{equation}
						W = W^*
					\end{equation}
				\item If she does not work
					\begin{equation}
						W^* \geq W
					\end{equation}
			\item Interpret hours of work as a slack variable
				\begin{equation}
					h \left( W^* - W \right) = 0
 				\end{equation}
			\end{itemize}
		\end{itemize}
\end{frame}	

\section{Estimation}
	\begin{frame}
		\frametitle{Estimation}
			\begin{itemize}
				\item Let $l(\cdot)$ be some strictly increasing transformation so that
					\begin{eqnarray}
						l(W_{i}^*) &=& \beta_{0} + \beta_{1}h_{i} + \beta_{2} W_{mi} + \beta_{3} P_{i} + \beta_{4} A_{i} + \beta_{5} Z_{i} + \varepsilon_{i} \nonumber \\
						l(W_{i}) &=& b_{0} + b_{1} S_{i} + b_{2} E_{i} + u_{i} 
					\end{eqnarray}
				\item If $W^* > W$ at zero hours of work, the reduced form equations for observed wages and hours is
				\begin{eqnarray}
				h_{i} &=& \frac{1}{\beta_{1}} ( b_{0} - \beta_{0} + b_{1} S_{i} + b_{2} E_{i} - \beta_{2} W_{mi} \nonumber \\
				&-& \beta_{3} P_{i} - \beta_{4} A_{i} - \beta_{3} Z_{i} ) + \frac{u_{i} - \varepsilon_{i}}{\beta_{1}} \nonumber \\
				l(W_{i}) &=& b_{0} + b_{1} S_{i} + b_{2} E_{i} + u_{i} 
				\end{eqnarray}
			\end{itemize}
	\end{frame}

\begin{frame}
	\frametitle{Estimation, contd 1}
		\begin{itemize}
			\item If there are $T$ women and $K$ work, it is possible to estimate the parameters of interest maximizing
			\begin{eqnarray}
				L(\cdot) &=& \prod _{i=1} ^{K} j \left( h_{i}, l(W_{i}) | (W_{i} > W_{i}^* )_{h = 0}\right) \cdot \Pr \left[ (W_{i} > W_{i}^* )_{h = 0} \right] \nonumber \\ 
				&\times& \prod _{i=K+1} ^{T} \Pr \left[ (W_{i} < W_{i}^* )_{h = 0} \right]
			\end{eqnarray}
		\end{itemize}
\noindent where $j(\cdot)$ is the conditional joint distribution of $h_{i}, l(W_{i})$
	\begin{itemize}
		\item How?
			\begin{enumerate}
				\item Assume that $\varepsilon, u_{i}$ are i.i.d. joint normal
				\item Use Yike's results on normality to simplify expressions (this is joint censored distribution)
			\end{enumerate}
	\end{itemize}
\end{frame}

\section{Extensions}

\begin{frame}
	\frametitle{Extensions}
		\begin{itemize}
			\item Gazillions of papers talk about selection
			\item One important extension is Heckman and Selacek (1985) which considers:
			\begin{itemize}
				\item Embed the Roy Model in a setting with
					\begin{enumerate}
						\item Non-market sector
						\item Employees select based on utility (instead of earnings)
						\item Permit unmeasured attributes to be log-normal
					\end{enumerate}
			\end{itemize}
			\item All the extensions help to better fit data of wage distribution from the U.S. labor market (CPS 1968-1981)
		\end{itemize}
\end{frame}

\end{document}