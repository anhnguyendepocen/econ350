\section{Discrete Choice Dynamic Programming}
Consider the dynamic version of the model in Section \ref{section:model}. The utility function, the budget constraint, and the distribution of the unobserved variables are the same. The dynamics of the model come through the wage process. In particular, $w_{it}$ increases with work experience, $h_{it}$. This equals the total number of periods that the woman in household $i$ accumulates in all the periods previous to $t$:
\begin{equation}
h_{it} = \sum\limits_{\tau=1}^{t-1} d_{i\tau},
\end{equation} 
\noindent where $h_{i1} = 0$ for simplicity. The wage function is
\begin{equation}
w_{it} = z_{i}\gamma_{1} + \gamma_{2}h_{it} + \eta_{it}.
\end{equation} 

\noindent For simplicity, the variables $\kappa_{it}$, $n_{it}$, $z_{it}$ are non-stochastic and time invariant.

\begin{exercise} (Dynamic Programming Set-up)
Write down the household's problem for each period $t$. Let $\delta$ be the discount factor, $\Omega_{it}$ the state space, and $\Omega_{it}^-$ the observed state space. Write down the elements in $\Omega_{it},\Omega_{it}^-$.\\
\noindent Answer:\\	
\noindent The problem is
\begin{equation}
\max_{d_{it}} \mathbb{E}\left\{ \sum\limits_{\tau=t}^{T}\delta^{\tau-t} \left[U^{1}_{i\tau}d_{i\tau} + U^{0}_{i\tau}\left( 1-d_{i\tau} \right)\right] \middle| \Omega_{it}\right\} 
\end{equation}

\noindent where
\begin{eqnarray}
U^{1}_{i\tau} &=& y_{i\tau} + z_{i}\gamma_{1} + \gamma_{2}h_{i\tau} + \eta_{i\tau} - \pi n_{i}\\
U^{0}_{i\tau} &=& y_{i\tau} + \beta_{\kappa}\kappa_{i} + \beta_{n}n_{i} + \epsilon_{i\tau}\\
\Omega_{it} &=& \left\{ y_{it}, z_{i}, n_{i}, \kappa_{i}, \epsilon_{it}, \eta_{it}, h_{it}\right\}\\
\Omega_{it}^- &=& \left\{ y_{it}, z_{i}, n_{i}, \kappa_{i}, h_{it}\right\}
\end{eqnarray}
\end{exercise}
\begin{exercise} (Bellman Equation)
Write down the recursive formulation of the household's problem.
\end{exercise}

\begin{exercise} (Solution)
Solve the dynamic problem of the household for for an arbitrary time horizon, $T$. Hint: use a backward recursion.
\end{exercise}

\subsection{Simulation and Estimation}

\begin{exercise} (Likelihood)
What is the individual likelihood of household $i$ at time $t$? What is the sample likelihood across all periods?
\end{exercise}

\begin{exercise} (Simulation) \label{exercise:simulation}
Simulate a balanced data set with $n = 1000$ observations and $T=6$. Use the same parameters as in the static model in Section \ref{section:models}. For the parameters that are exclusive of the dynamic model use the following: $\gamma_2 = 0.9,\delta = 0.85$. Set the experience of every woman to zero in $t=1$. Save the data in a ``.csv'' file.
\end{exercise}

\begin{exercise} (Estimation)
Estimate the parameters of the model by ML. Compare your results with the parameters in Exercise \ref{exercise:simulation}.
\end{exercise}