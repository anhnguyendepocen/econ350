%Input premable
\documentclass{beamer}

%Packages
\usepackage{graphicx}
\usepackage{graphics}
\usepackage{hyperref}
\usepackage[english]{babel}
\usepackage{amsmath}
\usepackage{amsfonts}
\usepackage{amssymb}
\usepackage{bm}
\usepackage{comment}
\usepackage{etoolbox}
\usepackage{graphicx}
\usepackage{tabularx,ragged2e,booktabs}
\usepackage{caption}
\usepackage{hyphenat}
\usepackage{fixltx2e}
\usepackage[para]{threeparttable}
\usepackage[capposition=top]{floatrow}
\usepackage{subcaption}
\usepackage{pdfpages}
\usepackage{natbib}
\usepackage{rotating}

%Math Functions
\DeclareMathOperator{\var}{Var}
\DeclareMathOperator{\cov}{Cov}

%Commands
\newcommand\independent{\protect\mathpalette{\protect\independenT}{\perp}}
\def\independenT#1#2{\mathrel{\rlap{$#1#2$}\mkern2mu{#1#2}}}
\newcommand{\overbar}[1]{\mkern 1.5mu\overline{\mkern-1.5mu#1\mkern-1.5mu}\mkern 1.5mu}
\newcommand{\equald}{\ensuremath{\overset{d}{=}}}

\newenvironment{wideitemize}{\itemize\addtolength{\itemsep}{10pt}}{\enditemize}

\mode<presentation>
	{
	\usetheme{UChicagoJorge}
	\setbeamercovered{transparent = 28}
	}

%\usecolortheme{UChicagoJorge} 
%\useinnertheme{UChicagoJorge}
%\useoutertheme{UChicagoJorge}

\title{Labor Supply Decisions}
\subtitle{Female vs. Male}
\author{Jorge L. Garc\'{i}a G.-Men\'{e}ndez}\institute{The University of Chicago, Economics}

\begin{document}

\begin{frame}[plain]
	\titlepage
\end{frame}


\AtBeginSection[]
{
   \begin{frame}
       \frametitle{Outline}
       \tableofcontents[currentsection]
   \end{frame}
}

\section{Two Models of Labor Supply}

\begin{frame}
	\frametitle{Objective}
		\begin{enumerate}
 			\item Objective: compare two models of life-cycle decisions
 			\begin{wideitemize}
 			\item One model for females one model for males
 			\item ``Females Model'': Keane and Wolpin (2010)
 			\item ``Males Model'': Keane and Wolpin (1997)
 			\end{wideitemize}
 			\item Learn about modeling decisions
 			\item Understand the main features of female and male life-cycle or career decisions
 		\end{enumerate}
\end{frame}

\section{Motivation of the Papers}
\begin{frame}
	\frametitle{Motivation: Females}
	\begin{enumerate}
		\item Large differences in economic and demographic characteristics of majority (white) compared to minority (black and Hispanic) women
		\item NLSY79 in 1990 (Ages 25 to 33):
		\begin{wideitemize}
			\item Mean schooling years: white 13.4; black 12.8; Hispanic 12.1 
			\item Marriage percentages: white 65\%; black 32\%; Hispanic 55\%.
			\item Children: white 1.2; black and Hispanic 1.7
			\item Employment: white 74\%, black 66\%, Hispanic 67\%
			\item AFDC previous year: white 4\%, black 20\%, Hispanic, 11\%  
		\end{wideitemize}
	\end{enumerate}
\end{frame}

\begin{frame}
	\frametitle{Motivation: Males}	
	\begin{enumerate}
		\item Analyze the ``life-cycle'' or career decisions of a core sample of white men
	\end{enumerate}
\end{frame}

\section{Research Question and Approach}
\begin{frame}
	\frametitle{Research Question and Approach: Females}
		\begin{wideitemize}
			\item Model labor supply, marriage markets, preference heterogeneity, and the welfare system to answer:
			\begin{enumerate}
			\item How much observed of observed minority-majority differences in behavior can attributed to differences in labor market and marriage opportunities, and preferences?
			\item How does to welfare system affects augment the differences minority-majority differences?
			\item How will the new cohorts that grow up under the new welfare system (TANF) behaves compared to older cohorts?
			\end{enumerate}
		\end{wideitemize}
		
\end{frame}

\begin{frame}
	\frametitle{Research Question and Approach: Males}
		\begin{wideitemize}
			\item Combine the extensions to the basic Roy (1951) model in Heckman and Sedlaeck (1985) and Willis (1986) to asses self-selection in three dimensions, schooling, work, and occupational choice, and understand
			\begin{enumerate}
				\item Human capital investment
				\item School attendance
				\item Work
				\item Occupational choices
				\item Future work decisions
				\item Wage patterns
			\end{enumerate}			 
		\end{wideitemize}
		
\end{frame}

\section{Models}


\section{Results}

\section{Comments}

\end{document}