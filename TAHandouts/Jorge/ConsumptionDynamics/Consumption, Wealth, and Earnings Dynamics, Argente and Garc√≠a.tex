
\documentclass[notes=show]{beamer}
%%%%%%%%%%%%%%%%%%%%%%%%%%%%%%%%%%%%%%%%%%%%%%%%%%%%%%%%%%%%%%%%%%%%%%%%%%%%%%%%%%%%%%%%%%%%%%%%%%%%%%%%%%%%%%%%%%%%%%%%%%%%%%%%%%%%%%%%%%%%%%%%%%%%%%%%%%%%%%%%%%%%%%%%%%%%%%%%%%%%%%%%%%%%%%%%%%%%%%%%%%%%%%%%%%%%%%%%%%%%%%%%%%%%%%%%%%%%%%%%%%%%%%%%%%%%
\usepackage{amsmath}
\usepackage{amssymb}
\usepackage{amsfonts}
\usepackage{mathpazo}
\usepackage{hyperref}
\usepackage{multimedia}

\setcounter{MaxMatrixCols}{10}
%TCIDATA{OutputFilter=LATEX.DLL}
%TCIDATA{Version=5.50.0.2953}
%TCIDATA{<META NAME="SaveForMode" CONTENT="1">}
%TCIDATA{BibliographyScheme=Manual}
%TCIDATA{Created=Wednesday, January 09, 2013 17:34:23}
%TCIDATA{LastRevised=Tuesday, January 29, 2013 12:52:20}
%TCIDATA{<META NAME="GraphicsSave" CONTENT="32">}
%TCIDATA{<META NAME="DocumentShell" CONTENT="Other Documents\SW\Slides - Beamer">}
%TCIDATA{Language=American English}
%TCIDATA{CSTFile=beamer.cst}

\newenvironment{stepenumerate}{\begin{enumerate}[<+->]}{\end{enumerate}}
\newenvironment{stepitemize}{\begin{itemize}[<+->]}{\end{itemize} }
\newenvironment{stepenumeratewithalert}{\begin{enumerate}[<+-| alert@+>]}{\end{enumerate}}
\newenvironment{stepitemizewithalert}{\begin{itemize}[<+-| alert@+>]}{\end{itemize} }
\usetheme{Madrid}
\input{tcilatex}
\begin{document}

\title[Consumption, Wealth, \& Earnings Dynamics]{Lifecycle Dynamics and
Inequality}
\subtitle{Consumption, Wealth, \& Earnings Dynamics}
\author[Argente \& Garc\'{\i}a]{David Argente \& Jorge Luis Garc\'{\i}a}
\institute{UChicago}
\date{\today }
\maketitle

\section{Structure of the Presentation}

%TCIMACRO{\TeXButton{BeginFrame}{\begin{frame}}}%
%BeginExpansion
\begin{frame}%
%EndExpansion

\QTR{frametitle}{Structure}

\begin{enumerate}
\item Broad Questions in this Literature

\item How Much Consumption Insurance beyond Self-Insurance? by Greg Kaplan
and Giovanni L. Violante.

\item Consumption and Labor Supply with Partial Insurance: an Analytical
Framework by Jonathan Heathcote, Kjetil Storesletten, \& Giovanni L.
Violante.

\item Consumption Inequality and Family Labor Supply by Richard Blundell,
Luigi Pistaferri, \& Itay Saporta-Eksten.

\item Final Comments.
\end{enumerate}

%TCIMACRO{\TeXButton{Transition: Box Out}{\transboxout}}%
%BeginExpansion
\transboxout%
%EndExpansion
%TCIMACRO{\TeXButton{EndFrame}{\end{frame}}}%
%BeginExpansion
\end{frame}%
%EndExpansion

\bigskip

\section{Broad Questions}

%TCIMACRO{\TeXButton{BeginFrame}{\begin{frame}}}%
%BeginExpansion
\begin{frame}%
%EndExpansion

\QTR{frametitle}{Broad Questions in this Literature that the Papers Tackle}

\QTR{framesubtitle}{Research Motivation of each Paper}

\begin{itemize}
\item The first paper addresses:
\end{itemize}

\begin{enumerate}
\item What is the pattern of consumption smoothing during the life cycle in
a context of incomplete markets.

\textit{Principal feature: contrast empirical findings to those of Blundell,
Pistaferri and Preston (BPP).}
\end{enumerate}

\begin{itemize}
\item The second paper addresses:
\end{itemize}

\begin{enumerate}
\item How effectively can households smooth idiosyncratic wage fluctuations
via private insurance arrangements and labor supply adjustments?

\item To what degree has the four-decade-long rise in wage dispersion passed
through to inequality in consumption and hours worked?

\item What is the role uninsurable life-cycles shocks to wages relative to
initial heterogeneity in skills and preferences in accounting for observed
inequality?
\end{enumerate}

%TCIMACRO{\TeXButton{Transition: Box Out}{\transboxout}}%
%BeginExpansion
\transboxout%
%EndExpansion
%TCIMACRO{\TeXButton{EndFrame}{\end{frame}}}%
%BeginExpansion
\end{frame}%
%EndExpansion

\bigskip

%TCIMACRO{\TeXButton{BeginFrame}{\begin{frame}}}%
%BeginExpansion
\begin{frame}%
%EndExpansion

\QTR{frametitle}{Broad Questions in this Literature that the Papers Tackle}

\QTR{framesubtitle}{Research Motivation of each Paper}

\begin{itemize}
\item The third paper addresses:
\end{itemize}

\begin{enumerate}
\item What is the link between wage inequality and consumption inequality?

\textit{Novel feature: a life cycle model incorporates household consumption
and family labor supply decision.}

\textit{Focus on the importance of family labor supply as an insurance
mechanism to wage shocks.}
\end{enumerate}

%TCIMACRO{\TeXButton{Transition: Box Out}{\transboxout}}%
%BeginExpansion
\transboxout%
%EndExpansion
%TCIMACRO{\TeXButton{EndFrame}{\end{frame}}}%
%BeginExpansion
\end{frame}%
%EndExpansion

\bigskip

\section{How Much Consumption Insurance beyond Self-Insurance?}

%TCIMACRO{\TeXButton{BeginFrame}{\begin{frame}}}%
%BeginExpansion
\begin{frame}%
%EndExpansion

\QTR{frametitle}{How Much Consumption Insurance beyond Self-Insurance? by
Greg Kaplan and Giovanni L. Violante.}

\textbf{Presentation in other file.}

%TCIMACRO{\TeXButton{Transition: Box Out}{\transboxout}}%
%BeginExpansion
\transboxout%
%EndExpansion
%TCIMACRO{\TeXButton{EndFrame}{\end{frame}}}%
%BeginExpansion
\end{frame}%
%EndExpansion

\bigskip 

\section{Consumption and Labor Supply with Partial Insurance: an Analytical
Framework}

%TCIMACRO{\TeXButton{BeginFrame}{\begin{frame}}}%
%BeginExpansion
\begin{frame}%
%EndExpansion

\QTR{frametitle}{Consumption and Labor Supply with Partial Insurance}

\bigskip 

\begin{center}
\textbf{Consumption and Labor Supply with Partial Insurance: an Analytical
Framework}

\textit{Jonathan Heathcote}

\textit{Kjetil Storesletten}

\textit{Giovanni L. Violante}
\end{center}

\bigskip 

%TCIMACRO{\TeXButton{Transition: Box Out}{\transboxout}}%
%BeginExpansion
\transboxout%
%EndExpansion
%TCIMACRO{\TeXButton{EndFrame}{\end{frame}}}%
%BeginExpansion
\end{frame}%
%EndExpansion

%TCIMACRO{\TeXButton{BeginFrame}{\begin{frame}}}%
%BeginExpansion
\begin{frame}%
%EndExpansion

\QTR{frametitle}{Consumption and Labor Supply with Partial Insurance}

\QTR{framesubtitle}{Model: Demographics and Preferences}

\begin{itemize}
\item Yaari perpetual youth model: agents born at age $0$ and survive from
age $a$ to age $a+1$ with constant probability $\delta <1$.

\item Lifetime utility \bigskip for agent born in cohort birth year $b$ is
given by%
\begin{equation*}
\mathbb{E}_{b}=\dsum\limits_{t=b}^{\infty }\left( \beta \delta \right)
^{t-b}u\left( c_{t},h_{t};\varphi \right) .
\end{equation*}
\end{itemize}

%TCIMACRO{\TeXButton{Transition: Box Out}{\transboxout}}%
%BeginExpansion
\transboxout%
%EndExpansion
%TCIMACRO{\TeXButton{EndFrame}{\end{frame}}}%
%BeginExpansion
\end{frame}%
%EndExpansion

\bigskip

%TCIMACRO{\TeXButton{BeginFrame}{\begin{frame}}}%
%BeginExpansion
\begin{frame}%
%EndExpansion

\QTR{frametitle}{Consumption and Labor Supply with Partial Insurance}

\QTR{framesubtitle}{Model: Period Utility and Heterogeneity Sources}

\begin{itemize}
\item The period utility is%
\begin{equation}
u\left( c_{t},h_{t};\varphi \right) =\frac{c_{t}^{1-\gamma }-1}{1-\gamma }%
-\exp \left( \varphi \right) \frac{h_{t}^{1+\sigma }}{1+\sigma }
\end{equation}

\item The preference "weight" $\varphi $ captures the strength of an
individual's aversion to work: its distribution for the cohort with birth
year $t$ is denoted by $F_{\varphi t}$ (with cohort specific variance $%
\upsilon _{\varphi t}$).

\item This source of heterogeneity helps explaining the observed
cross-sectional joint distribution over wages, hours, and consumption.
\end{itemize}

%TCIMACRO{\TeXButton{Transition: Box Out}{\transboxout}}%
%BeginExpansion
\transboxout%
%EndExpansion
%TCIMACRO{\TeXButton{EndFrame}{\end{frame}}}%
%BeginExpansion
\end{frame}%
%EndExpansion

\bigskip

%TCIMACRO{\TeXButton{BeginFrame}{\begin{frame}}}%
%BeginExpansion
\begin{frame}%
%EndExpansion

\QTR{frametitle}{Consumption and Labor Supply with Partial Insurance}

\QTR{framesubtitle}{Model: Idiosyncratic Risk}

\begin{itemize}
\item The population in the economy is partitioned into "islands", where
each island contains a continuum of individuals.

\item Agents face two types of orthogonal, labor productivity shocks

\begin{enumerate}
\item At the individual level: uncorrelated across island members and
denoted by $\varepsilon _{t}$.

\item At island level: uncorrelated across islands and denoted by $\alpha
_{t}$.
\end{enumerate}

\item Individual labor productivity, $w_{t}$ has the following log process%
\begin{equation}
\log w_{t}=\alpha _{t}+\varepsilon _{t}.
\end{equation}
\end{itemize}

%TCIMACRO{\TeXButton{Transition: Box Out}{\transboxout}}%
%BeginExpansion
\transboxout%
%EndExpansion
%TCIMACRO{\TeXButton{EndFrame}{\end{frame}}}%
%BeginExpansion
\end{frame}%
%EndExpansion

\bigskip

%TCIMACRO{\TeXButton{BeginFrame}{\begin{frame}}}%
%BeginExpansion
\begin{frame}%
%EndExpansion

\QTR{frametitle}{Consumption and Labor Supply with Partial Insurance}

\QTR{framesubtitle}{Model: Shock Processes}

\begin{itemize}
\item Island-level component%
\begin{equation}
\alpha _{t}=\alpha _{t-1}+\omega _{t}.
\end{equation}

\item Individual-level component%
\begin{eqnarray}
\varepsilon _{t} &=&\varkappa _{t}+\theta _{t} \\
\varkappa _{t} &=&\varkappa _{t-1}+\eta _{t}
\end{eqnarray}

where $\theta _{t}$ is an i.i.d. transitory component ; $\varkappa _{t}$ is
the permanent component and follows a unit root process.

\item Agents who enter the labor market at age $a=0$ in year $t$ draw
initial realizations from cohort specific distributions $F_{\alpha
^{0}t},F_{\varkappa ^{0}t}$. The initial draws $\alpha ^{0},\varkappa
^{0},\varphi $ are uncorrelated.
\end{itemize}

%TCIMACRO{\TeXButton{Transition: Box Out}{\transboxout}}%
%BeginExpansion
\transboxout%
%EndExpansion
%TCIMACRO{\TeXButton{EndFrame}{\end{frame}}}%
%BeginExpansion
\end{frame}%
%EndExpansion

\bigskip

%TCIMACRO{\TeXButton{BeginFrame}{\begin{frame}}}%
%BeginExpansion
\begin{frame}%
%EndExpansion

\QTR{frametitle}{Consumption and Labor Supply with Partial Insurance}

\QTR{framesubtitle}{Model: Production and Taxes}

\begin{itemize}
\item One final good; CRS technology; consumption and labor are traded in a
perfectly competitive economy; wages are equal to individual productivities.

\item Progressive tax system. Following Benabou (2002), an individual with
gross labor income $y_{t}=w_{t}h_{t}$ receives disposable post-government
earnings given by 
\begin{equation}
\overset{\thicksim }{y_{t}}=\lambda \left( y_{t}\right) ^{1-\tau }.
\end{equation}
\end{itemize}

%TCIMACRO{\TeXButton{Transition: Box Out}{\transboxout}}%
%BeginExpansion
\transboxout%
%EndExpansion
%TCIMACRO{\TeXButton{EndFrame}{\end{frame}}}%
%BeginExpansion
\end{frame}%
%EndExpansion

\bigskip

%TCIMACRO{\TeXButton{BeginFrame}{\begin{frame}}}%
%BeginExpansion
\begin{frame}%
%EndExpansion

\QTR{frametitle}{Consumption and Labor Supply with Partial Insurance}

\QTR{framesubtitle}{Model: Market Structure, Insurance, and Information}

\begin{itemize}
\item All assets in the economy are in zero net supply; the asset market is
competitive; agents are endowed with zero initial wealth.

\item Trade structure:

\begin{enumerate}
\item Initial draws happen before markets open.

\item Island-location is defined; each island is characterized by an ex-ante
unknown sequence $\left\{ \omega _{t}\right\} _{t=b+1}^{\infty }$ that
applies to all island members.

\item Within island, agents trade a complete set of insurance contracts: at
every period $t\geq b$ agents can purchase contracts indexed to $%
s_{t+1}=\left( \omega _{t+1},\eta _{t+1},\theta _{t+1}\right) $.

\item Across island, agents only trade insurance contracts indexed to their
individual-level shocks $\left( \eta _{t+1},\theta _{t+1}\right) $.
Inter-island contracts contingent on the realization of the island-level
shock $\omega _{t+1}$ are ruled out.
\end{enumerate}
\end{itemize}

%TCIMACRO{\TeXButton{Transition: Box Out}{\transboxout}}%
%BeginExpansion
\transboxout%
%EndExpansion
%TCIMACRO{\TeXButton{EndFrame}{\end{frame}}}%
%BeginExpansion
\end{frame}%
%EndExpansion

\bigskip

%TCIMACRO{\TeXButton{BeginFrame}{\begin{frame}}}%
%BeginExpansion
\begin{frame}%
%EndExpansion

\QTR{frametitle}{Consumption and Labor Supply with Partial Insurance}

\QTR{framesubtitle}{Model: Market Structure, Insurance, and Information
(continued...)}

\begin{itemize}
\item Insurance:

\begin{enumerate}
\item Insurance contracts incorporate mortality risk (when state $s_{t+1}$
is insured against and realized the contracts pay $\frac{1}{\delta }$ if the
agents survives and $0$ if she dies) .
\end{enumerate}

\item Information: agents take as given the sequences of distributions $%
\left\{ F_{\varphi t},F_{\alpha ^{0}t},F_{\varkappa ^{0}t},F_{\omega
t},F_{\eta t},F_{\theta t}\right\} $, i.e. they have perfect foresight over
future wage distributions.
\end{itemize}

%TCIMACRO{\TeXButton{Transition: Box Out}{\transboxout}}%
%BeginExpansion
\transboxout%
%EndExpansion
%TCIMACRO{\TeXButton{EndFrame}{\end{frame}}}%
%BeginExpansion
\end{frame}%
%EndExpansion

\bigskip

%TCIMACRO{\TeXButton{BeginFrame}{\begin{frame}}}%
%BeginExpansion
\begin{frame}%
%EndExpansion

\QTR{frametitle}{Consumption and Labor Supply with Partial Insurance}

\QTR{framesubtitle}{Model: Agent's Problem}

\begin{itemize}
\item Let $s^{t}=\left( s_{b},s_{b+1},\ldots ,s_{t}\right) $ denote the
individual history of the shocks for an agent from birth year $b$ up to date 
$t$, where%
\begin{equation}
s_{j}=\left\{ 
\begin{array}{ccc}
\left( b,\varphi ,\alpha ^{0},\varkappa ^{0},\theta _{b}\right) & \in {%
\mathbb{S}}_b=%
%TCIMACRO{\U{2115} }%
%BeginExpansion
\mathbb{N}
%EndExpansion
\times 
%TCIMACRO{\U{211d} }%
%BeginExpansion
\mathbb{R}
%EndExpansion
^{4} & j=b \\ 
\left( \omega _{j},\eta _{j},\theta _{j}\right) & \in \mathbb{S}=%
%TCIMACRO{\U{211d} }%
%BeginExpansion
\mathbb{R}
%EndExpansion
^{3} & j\neq b%
\end{array}%
\right.
\end{equation}

with $s^{t}={\mathbb{S}}_{b}\times \mathbb{S}^{t-b}$.

\item Also, define $z_{t+1}=$ $\left( \eta _{t+1},\theta _{t+1}\right) \in
Z\subseteq 
%TCIMACRO{\U{2124} }%
%BeginExpansion
\mathbb{Z}
%EndExpansion
=%
%TCIMACRO{\U{211d} }%
%BeginExpansion
\mathbb{R}
%EndExpansion
^{2}$.
\end{itemize}

%TCIMACRO{\TeXButton{Transition: Box Out}{\transboxout}}%
%BeginExpansion
\transboxout%
%EndExpansion
%TCIMACRO{\TeXButton{EndFrame}{\end{frame}}}%
%BeginExpansion
\end{frame}%
%EndExpansion

\bigskip

%TCIMACRO{\TeXButton{BeginFrame}{\begin{frame}}}%
%BeginExpansion
\begin{frame}%
%EndExpansion

\QTR{frametitle}{Consumption and Labor Supply with Partial Insurance}

\QTR{framesubtitle}{Model: Agent's Problem (continued...)}

\begin{itemize}
\item The budget constraint is%
\begin{eqnarray}
\lambda \left[ w_{t}\left( s^{t}\right) h_{t}\left( s^{t}\right) \right]
^{1-\tau }+d\left( s^{t}\right) -c_{t}\left( s^{t}\right) &=&\dint\limits_{%
\mathbb{S}}Q_{t}\left( \cdot \right) B_{t}\left( \cdot \right) ds_{t+1} \\
&&+\dint\limits_{%
%TCIMACRO{\U{2124} }%
%BeginExpansion
\mathbb{Z}
%EndExpansion
}Q_{t}^{\ast }\left( \cdot \right) B_{t}^{\ast }\left( \cdot \right) dz_{t+1}
\notag
\end{eqnarray}

where 
\begin{equation}
d\left( s^{t}\right) =\delta ^{-1}\left[ B_{t}\left( s_{t};s^{t-1}\right)
+B_{t}^{\ast }\left( z_{t};s^{t-1}\right) \right] .
\end{equation}
\end{itemize}

%TCIMACRO{\TeXButton{Transition: Box Out}{\transboxout}}%
%BeginExpansion
\transboxout%
%EndExpansion
%TCIMACRO{\TeXButton{EndFrame}{\end{frame}}}%
%BeginExpansion
\end{frame}%
%EndExpansion

\bigskip

%TCIMACRO{\TeXButton{BeginFrame}{\begin{frame}}}%
%BeginExpansion
\begin{frame}%
%EndExpansion

\QTR{frametitle}{Consumption and Labor Supply with Partial Insurance}

\QTR{framesubtitle}{Model: Competitive Equilibrium}

\begin{itemize}
\item There exists a competitive equilibrium in which%
\begin{eqnarray}
B_{t}^{\ast }\left( Z,s^{t}\right) &=&0\forall Z,s^{t} \\
\log c_{t} &=&-\left( 1-\tau \right) \overset{\symbol{94}}{\varphi }+\left(
1-\tau \right) \left( \frac{1+\overset{\symbol{94}}{\sigma }}{\overset{%
\symbol{94}}{\sigma }+\gamma }\right) \alpha _{t}+C_{t}^{a}  \notag \\
\log h_{t} &=&-\overset{\symbol{94}}{\varphi }+\left( \frac{1-\gamma }{%
\overset{\symbol{94}}{\sigma }+\gamma }\right) \alpha _{t}+\frac{1}{\overset{%
\symbol{94}}{\sigma }}\varepsilon _{t}+H_{t}^{a}.  \notag
\end{eqnarray}
\end{itemize}

%TCIMACRO{\TeXButton{Transition: Box Out}{\transboxout}}%
%BeginExpansion
\transboxout%
%EndExpansion
%TCIMACRO{\TeXButton{EndFrame}{\end{frame}}}%
%BeginExpansion
\end{frame}%
%EndExpansion

\bigskip

%TCIMACRO{\TeXButton{BeginFrame}{\begin{frame}}}%
%BeginExpansion
\begin{frame}%
%EndExpansion

\QTR{frametitle}{Consumption and Labor Supply with Partial Insurance}

\QTR{framesubtitle}{Model: Income vs. Substitution Effects}

\begin{itemize}
\item A higher $\varphi $ reduces labor: higher distaste for work. This
transmits in a straightforward way to earnings and consumption.

\item Hours worked are increasing in the insurable component: and its
response to $\varepsilon _{t}$ is given by $\frac{1}{\overset{\symbol{94}}{%
\sigma }}$. (perfect insurance rules out income effect)

\item Hours worked ambiguously change after an uninsurable shock: if $\gamma
>1$ the income effect dominates and hours worked fall after an increase in $%
\alpha _{t}$. The converse happens when $\gamma <1$.
\end{itemize}

%TCIMACRO{\TeXButton{Transition: Box Out}{\transboxout}}%
%BeginExpansion
\transboxout%
%EndExpansion
%TCIMACRO{\TeXButton{EndFrame}{\end{frame}}}%
%BeginExpansion
\end{frame}%
%EndExpansion

\bigskip

%TCIMACRO{\TeXButton{BeginFrame}{\begin{frame}}}%
%BeginExpansion
\begin{frame}%
%EndExpansion

\QTR{frametitle}{Consumption and Labor Supply with Partial Insurance}

\QTR{framesubtitle}{Model: Competitive Equilibrium (continued...)}

\begin{itemize}
\item This equilibrium implies zero private insurance against shocks $\alpha
_{t}$: this shocks are common to all island members and there is no trade
across islands. (\textit{uninsurable})

\item There is perfect insurance against shocks to $\varepsilon _{t}$: this
shocks are "washed-out" within islands (\textit{insurable}).
\end{itemize}

%TCIMACRO{\TeXButton{Transition: Box Out}{\transboxout}}%
%BeginExpansion
\transboxout%
%EndExpansion
%TCIMACRO{\TeXButton{EndFrame}{\end{frame}}}%
%BeginExpansion
\end{frame}%
%EndExpansion

\bigskip

%TCIMACRO{\TeXButton{BeginFrame}{\begin{frame}}}%
%BeginExpansion
\begin{frame}%
%EndExpansion

\QTR{frametitle}{Consumption and Labor Supply with Partial Insurance}

\QTR{framesubtitle}{Model: Tractability, Insurance Dichotomy, and the
Constantinides-Duffie Condition}

\begin{itemize}
\item Usually, incomplete markets models do not admit an analytical solution
(need to use numerical methods). This paper retains tractability because:

\begin{enumerate}
\item Individual wealth is a redundant variable.

\item Agents have access to perfect private insurance against some shocks.
\end{enumerate}

\item Why does this help?

\begin{enumerate}
\item Full-insurance within island implies that within-island allocations
can be derived by solving a planner problem with a weighted function defined
by the initial asset positions for all agents, subject to an island-level
budget constraint (and then appeal to the Second Theorem of Welfare
Economics and support this allocation with a particular vector of prices).

\item The inter-island wealth distribution does not show up in the
allocations because it remains degenerate at zero.
\end{enumerate}
\end{itemize}

%TCIMACRO{\TeXButton{Transition: Box Out}{\transboxout}}%
%BeginExpansion
\transboxout%
%EndExpansion
%TCIMACRO{\TeXButton{EndFrame}{\end{frame}}}%
%BeginExpansion
\end{frame}%
%EndExpansion

\bigskip

%TCIMACRO{\TeXButton{BeginFrame}{\begin{frame}}}%
%BeginExpansion
\begin{frame}%
%EndExpansion

\QTR{frametitle}{Consumption and Labor Supply with Partial Insurance}

\QTR{framesubtitle}{Model: Tractability, Insurance Dichotomy, and the
Constantinides-Duffie Condition (continued...)}

\begin{itemize}
\item The island is a technical structure that allows agents to

\begin{enumerate}
\item Trade an unrestricted set of insurance claims: perfect insurance
against $\varepsilon _{t}$ is possible.

\item Have no inter-island trade: agents cannot pool the island-level risk.
\end{enumerate}

\item This is what defines the "partial insurability" that characterizes the
model.

\item Some permanent shocks are perfectly insured: "excess of smoothing"
(contradicting the permanent income hypothesis as observed ). This is
actually a feature of the data which Krueger (2006), Attanasio and Pavoni
(2006), and others have modeled through incomplete "partial insurance
models".
\end{itemize}

%TCIMACRO{\TeXButton{Transition: Box Out}{\transboxout}}%
%BeginExpansion
\transboxout%
%EndExpansion
%TCIMACRO{\TeXButton{EndFrame}{\end{frame}}}%
%BeginExpansion
\end{frame}%
%EndExpansion

\bigskip 

%TCIMACRO{\TeXButton{BeginFrame}{\begin{frame}}}%
%BeginExpansion
\begin{frame}%
%EndExpansion

\QTR{frametitle}{Consumption and Labor Supply with Partial Insurance}

\QTR{framesubtitle}{Model: Tractability, Insurance Dichotomy, and the
Constantinides-Duffie Condition (continued...)}

\begin{itemize}
\item This theoretical framework has as an important forebear Constantinides
and Duffie (CD, 1996).

\item CD model an environment in which no-trade equilibrium exists when:

\begin{enumerate}
\item Income process is multiplicative and $I\left( 1\right) $.

\item Innovations are drawn from a common distribution. 

\item Preferences are in the power utility class. 

\item Assets are in zero net supply.
\end{enumerate}

\item Implications: no risk sharing/each individual consumes her own initial
endowment.
\end{itemize}

%TCIMACRO{\TeXButton{Transition: Box Out}{\transboxout}}%
%BeginExpansion
\transboxout%
%EndExpansion
%TCIMACRO{\TeXButton{EndFrame}{\end{frame}}}%
%BeginExpansion
\end{frame}%
%EndExpansion

\bigskip 

%TCIMACRO{\TeXButton{BeginFrame}{\begin{frame}}}%
%BeginExpansion
\begin{frame}%
%EndExpansion

\QTR{frametitle}{Consumption and Labor Supply with Partial Insurance}

\QTR{framesubtitle}{Model: Tractability, Insurance Dichotomy, and the
Constantinides-Duffie Condition (continued...)}

\begin{itemize}
\item This model extends the CD environment to a richer framework:

\begin{enumerate}
\item Primitive exogenous stochastic process is over hourly wages and
includes a transitory component beyond the unit root.

\item Allow for progressive taxation: distinguish between government
provided insurance and private insurance.

\item Agents are heterogeneous with respect to work distaste.

\item Risks are privately insurable within islands, so the version of "no
trade" applies across islands rather across individuals. 
\end{enumerate}

\item One interpretation of this model is that the CD holds at island-level:
there is no risk sharing between islands, although there is private
insurance at the individual level. Again: this gives rise to partial
insurance.
\end{itemize}

%TCIMACRO{\TeXButton{Transition: Box Out}{\transboxout}}%
%BeginExpansion
\transboxout%
%EndExpansion
%TCIMACRO{\TeXButton{EndFrame}{\end{frame}}}%
%BeginExpansion
\end{frame}%
%EndExpansion

\bigskip 

%TCIMACRO{\TeXButton{BeginFrame}{\begin{frame}}}%
%BeginExpansion
\begin{frame}%
%EndExpansion

\QTR{frametitle}{Consumption and Labor Supply with Partial Insurance}

\QTR{framesubtitle}{Summary of Notation}

\begin{itemize}
\item Before going on with the identification strategy, consider the
following summary of notation:

\begin{itemize}
\item $c_{t},h\,_{t}$ endogenous variables: consumption and hours worked.

\item $y_{t},w_{t}$ earnings and wage related by $y_{t}=w_{t}h_{t}$.

\item $\delta ,\tau $ are the parameters set outside of the model:
probability of dying and tax rate.

\item $\gamma ,\sigma $ EIS: consumption and hours worked.

\item $\alpha _{t}$ island level component of the shock (only permanent with
shock $\omega _{t}$). 

\item $\varepsilon _{t}$ individual level component of the shock (permanent $%
\varkappa _{t}$ with shock $\eta _{t}$ + transitory $\theta _{t}$).

\item $\frac{1}{\overset{\symbol{94}}{\sigma }}\equiv \frac{1-\tau }{\sigma
+\tau },\overset{\symbol{94}}{\varphi }\equiv \varphi /\left( \sigma +\gamma
+\tau \left( 1-\gamma \right) \right) $ tax adjusted parameters.

\item  $\upsilon _{\mu h},\upsilon _{\mu y},\upsilon _{\mu c}$ variances of
the measurement error (stationary).

\item $\upsilon _{\overset{\symbol{94}}{\varphi }t},\upsilon _{\alpha
^{0}t},\upsilon _{\varkappa ^{0}t},\upsilon _{\theta t},\upsilon _{\omega
t},\upsilon _{\eta t}$: variance of shock 
\end{itemize}
\end{itemize}

%TCIMACRO{\TeXButton{Transition: Box Out}{\transboxout}}%
%BeginExpansion
\transboxout%
%EndExpansion
%TCIMACRO{\TeXButton{EndFrame}{\end{frame}}}%
%BeginExpansion
\end{frame}%
%EndExpansion

\bigskip

%TCIMACRO{\TeXButton{BeginFrame}{\begin{frame}}}%
%BeginExpansion
\begin{frame}%
%EndExpansion

\QTR{frametitle}{Consumption and Labor Supply with Partial Insurance}

\QTR{framesubtitle}{Identification and Estimation Method}

\begin{itemize}
\item The baseline scenario of "data needs" is an unbalanced panel on wages
and hours (e.g., PSID) and a repeated cross section on wages, hours, and
consumption (e.g., CEX).

\item The authors show that the parameters $\left\{ \sigma ,\gamma ,\upsilon
_{\mu h},\upsilon _{\mu y},\upsilon _{\mu c}\right\} $ as well as the
sequences $\left\{ \upsilon _{\overset{\symbol{94}}{\varphi }t},\upsilon
_{\alpha ^{0}t}\right\} _{t=1}^{T},\left\{ \upsilon _{\varkappa
^{0}t},\upsilon _{\theta t}\right\} _{t=1}^{T},\left\{ \upsilon _{\omega
t}\right\} _{t=2}^{T},\left\{ \upsilon _{\eta t}\right\} _{t=2}^{T}$ are
identified. Also, the identify $\upsilon _{\eta T}+\upsilon _{\theta T}$ and 
$\upsilon _{\varkappa ^{0}T}+\upsilon _{\theta T}$.

\item Other variations of identification are considered due to data
constraints (CEX and PSID differ in time availability, PSID becomes
biannual, etc.).

\item The authors show that with an external measure of measurement error in
earnings, $\upsilon _{\mu y}$, all the parameters could be identified
without using consumption data.
\end{itemize}

%TCIMACRO{\TeXButton{Transition: Box Out}{\transboxout}}%
%BeginExpansion
\transboxout%
%EndExpansion
%TCIMACRO{\TeXButton{EndFrame}{\end{frame}}}%
%BeginExpansion
\end{frame}%
%EndExpansion

\bigskip

%TCIMACRO{\TeXButton{BeginFrame}{\begin{frame}}}%
%BeginExpansion
\begin{frame}%
%EndExpansion

\QTR{frametitle}{Consumption and Labor Supply with Partial Insurance}

\QTR{framesubtitle}{Identification and Estimation Method (continued...)}

\begin{itemize}
\item We discuss the baseline Identification.

\item Use within-cohort "macro" moments to identify $\overset{\symbol{94}}{%
\sigma },\gamma ,\left\{ \upsilon _{\omega t}\right\} _{t=2}^{T},\left\{
\upsilon _{\eta t}+\Delta \upsilon _{\theta t}\right\} _{t=2}^{T}$.

\begin{itemize}
\item $\upsilon _{\omega t}=\Delta cov_{t}^{a}\left( \log \overset{\symbol{94%
}}{w},\log \overset{\symbol{94}}{c}\right) ^{2}/\Delta var_{t}^{a}\left(
\log \overset{\symbol{94}}{c}\right) $

\item $\upsilon _{\eta t}+\Delta \upsilon _{\theta t}=\Delta
var_{t}^{a}\left( \log \overset{\symbol{94}}{w}\right) $

\item $\Delta var_{t}^{a}\left( \log \overset{\symbol{94}}{h}\right) =\left[
\Delta cov_{t}^{a}\left( \log \overset{\symbol{94}}{h},\log \overset{\symbol{%
94}}{c}\right) /\Delta cov_{t}^{a}\left( \log \overset{\symbol{94}}{w},\log 
\overset{\symbol{94}}{c}\right) \right] ^{2}\upsilon _{\omega t}+\frac{1}{%
\overset{\symbol{94}}{\sigma }^{2}}\left( \upsilon _{\eta t}+\Delta \upsilon
_{\theta t}\right) $ to pick-up $\overset{\symbol{94}}{\sigma }$.

\item $\Delta cov_{t}^{a}\left( \log \overset{\symbol{94}}{h},\log \overset{%
\symbol{94}}{c}\right) /\Delta cov_{t}^{a}\left( \log \overset{\symbol{94}}{w%
},\log \overset{\symbol{94}}{c}\right) =\frac{1-\gamma }{\overset{\symbol{94}%
}{\sigma }+\gamma }$ to pick-up $\gamma $.
\end{itemize}
\end{itemize}

%TCIMACRO{\TeXButton{Transition: Box Out}{\transboxout}}%
%BeginExpansion
\transboxout%
%EndExpansion
%TCIMACRO{\TeXButton{EndFrame}{\end{frame}}}%
%BeginExpansion
\end{frame}%
%EndExpansion

\bigskip

%TCIMACRO{\TeXButton{BeginFrame}{\begin{frame}}}%
%BeginExpansion
\begin{frame}%
%EndExpansion

\QTR{frametitle}{Consumption and Labor Supply with Partial Insurance}

\QTR{framesubtitle}{Identification and Estimation Method (continued...)}

\begin{itemize}
\item Use the difference between the dispersion in growth rates ("micro
moments") and the growth rates of dispersion ("macro" moments) to identify $%
\left\{ \upsilon _{\theta t}\right\} _{t=1}^{T}$:

\begin{itemize}
\item $cov_{t}^{a}\left( \Delta \log \overset{\symbol{94}}{w},\Delta \log 
\overset{\symbol{94}}{c}\right) +var_{t}^{a}\left( \Delta \log \overset{%
\symbol{94}}{h}\right) -\Delta cov_{t}^{a}\left( \log \overset{\symbol{94}}{w%
},\log \overset{\symbol{94}}{c}\right) -\Delta var_{t}^{a}\left( \log 
\overset{\symbol{94}}{h}\right) =2\left( 1+\overset{\symbol{94}}{\sigma }%
\right) \frac{1}{\overset{\symbol{94}}{\sigma }^{2}\upsilon _{\theta t}}$
\end{itemize}

\item Combine $\left\{ \upsilon _{\theta t}\right\} _{t=1}^{T}$ with $%
\left\{ \upsilon _{\eta t}+\Delta \upsilon _{\theta t}\right\} _{t=2}^{T}$
to identify $\left\{ \upsilon _{\eta t}\right\} _{t=2}^{T}$, substitute $%
\upsilon _{\theta T-1}$ into $\left( \upsilon _{\eta T}+\Delta \upsilon
_{\theta T}\right) $ and from the first step identify $\left( \upsilon
_{\eta T}+\upsilon _{\theta T}\right) $.

\item Use initial covariances to obtain $\left\{ \upsilon _{\overset{\symbol{%
94}}{\varphi }t},\upsilon _{\alpha ^{0}t}\right\} _{t=1}^{T}$:

\begin{itemize}
\item $cov_{t}^{0}\left( \log \overset{\symbol{94}}{w},\log \overset{\symbol{%
94}}{c}\right) =\left( 1-\tau \right) \left( 1+\overset{\symbol{94}}{\sigma }%
\right) /\left( \overset{\symbol{94}}{\sigma }+\gamma \right) \upsilon
_{\alpha ^{0}t}$.
\end{itemize}
\end{itemize}

%TCIMACRO{\TeXButton{Transition: Box Out}{\transboxout}}%
%BeginExpansion
\transboxout%
%EndExpansion
%TCIMACRO{\TeXButton{EndFrame}{\end{frame}}}%
%BeginExpansion
\end{frame}%
%EndExpansion

\bigskip

%TCIMACRO{\TeXButton{BeginFrame}{\begin{frame}}}%
%BeginExpansion
\begin{frame}%
%EndExpansion

\QTR{frametitle}{Consumption and Labor Supply with Partial Insurance}

\QTR{framesubtitle}{Identification and Estimation Method (continued...)}

\begin{itemize}
\item ...

\begin{itemize}
\item $cov_{t}^{0}\left( \log \overset{\symbol{94}}{h},\log \overset{\symbol{%
94}}{c}\right) =\left( 1-\tau \right) \upsilon _{\overset{\symbol{94}}{%
\varphi }t}+\left( 1-\tau \right) \left( 1+\overset{\symbol{94}}{\sigma }%
\right) \left( 1-\gamma \right) /\left( \overset{\symbol{94}}{\sigma }%
+\gamma \right) ^{2}\upsilon _{\alpha ^{0}t}$.
\end{itemize}

\item Use initial variances to obtain $\left\{ \upsilon _{\varkappa
^{0}t}\right\} _{t=1}^{T-1}$ and $\left( \upsilon _{\varkappa
^{0}T}+\upsilon _{\theta T}\right) $:

\begin{itemize}
\item $cov_{t}^{0}\left( \log \overset{\symbol{94}}{h},\log \overset{\symbol{%
94}}{w}\right) +var_{t}^{0}\left( \log \overset{\symbol{94}}{h}\right)
=\upsilon _{\overset{\symbol{94}}{\varphi }t}+\left( 1-\gamma \right) \left(
1+\overset{\symbol{94}}{\sigma }\right) /\left( \overset{\symbol{94}}{\sigma 
}+\gamma \right) ^{2}\upsilon _{\alpha ^{0}t}+\left( 1+\overset{\symbol{94}}{%
\sigma }\right) /\overset{\symbol{94}}{\sigma }^{2}\upsilon _{\varkappa
^{0}t}+\upsilon _{\theta t}$.
\end{itemize}

\item And finally use $cov_{t}^{0}\left( \log \overset{\symbol{94}}{h},\log 
\overset{\symbol{94}}{w}\right) ,var_{t}^{0}\left( \log \overset{\symbol{94}}%
{w}\right) ,var_{t}^{0}\left( \log \overset{\symbol{94}}{c}\right) $ to
identify the measurement error parameters.
\end{itemize}

%TCIMACRO{\TeXButton{Transition: Box Out}{\transboxout}}%
%BeginExpansion
\transboxout%
%EndExpansion
%TCIMACRO{\TeXButton{EndFrame}{\end{frame}}}%
%BeginExpansion
\end{frame}%
%EndExpansion

\bigskip

%TCIMACRO{\TeXButton{BeginFrame}{\begin{frame}}}%
%BeginExpansion
\begin{frame}%
%EndExpansion

\QTR{frametitle}{Consumption and Labor Supply with Partial Insurance}

\QTR{framesubtitle}{Identification and Estimation Method (continued...)}

\begin{itemize}
\item Use male population between ages $25$ and $59$ who work at least $260$
hours in the year (avoid selection problems).

\item Regress all variables against year dummies, quartic in age, and (for
consumption) household composition dummies.

\item The authors minimize the weighted sum of differences between each
moment in the data and its empirical counterpart.

\item They pick the moments according to an identity weighting matrix and
use block-bootstrap at the household level to estimate standard errors.

\item Parameters set outside the model: $\delta =.996$ to match the
annualized US data; regress $\overset{\thicksim }{y_{t}}=\lambda \left(
y_{t}\right) ^{1-\tau }$ in logs to get a consistent estimate of $\left(
1-\tau \right) $.
\end{itemize}

%TCIMACRO{\TeXButton{Transition: Box Out}{\transboxout}}%
%BeginExpansion
\transboxout%
%EndExpansion
%TCIMACRO{\TeXButton{EndFrame}{\end{frame}}}%
%BeginExpansion
\end{frame}%
%EndExpansion

\bigskip

%TCIMACRO{\TeXButton{BeginFrame}{\begin{frame}}}%
%BeginExpansion
\begin{frame}%
%EndExpansion

\QTR{frametitle}{Consumption and Labor Supply with Partial Insurance}

\QTR{framesubtitle}{Results}

\begin{center}
\textbf{\FRAME{dtbpF}{4.7573in}{2.3385in}{0pt}{}{}{Figure}{\special{language
"Scientific Word";type "GRAPHIC";maintain-aspect-ratio TRUE;display
"USEDEF";valid_file "T";width 4.7573in;height 2.3385in;depth
0pt;original-width 14.3775in;original-height 7.043in;cropleft "0";croptop
"1";cropright "1";cropbottom "0";tempfilename
'MHEBXB00.wmf';tempfile-properties "XPR";}}}
\end{center}

%TCIMACRO{\TeXButton{Transition: Box Out}{\transboxout}}%
%BeginExpansion
\transboxout%
%EndExpansion
%TCIMACRO{\TeXButton{EndFrame}{\end{frame}}}%
%BeginExpansion
\end{frame}%
%EndExpansion

\bigskip

%TCIMACRO{\TeXButton{BeginFrame}{\begin{frame}}}%
%BeginExpansion
\begin{frame}%
%EndExpansion

\QTR{frametitle}{Consumption and Labor Supply with Partial Insurance}

\QTR{framesubtitle}{Results (continued...)}

\begin{itemize}
\item Implied Frisch elasticity of $\frac{1}{\overset{\symbol{94}}{\sigma }}%
=.38$, in the ballpark of the literature (Keane, 2011)

\item $38\%$ of permanent life-cycle wage innovations are insurable; $30\%$
of wage variation at labor market entry is insurable.

\item Transitory shocks are more variable.

\item Figures 1 and 2 show that the model-implied moments almost always lie
within the $90-10$ confidence intervals around the empirical moments.
\end{itemize}

%TCIMACRO{\TeXButton{Transition: Box Out}{\transboxout}}%
%BeginExpansion
\transboxout%
%EndExpansion
%TCIMACRO{\TeXButton{EndFrame}{\end{frame}}}%
%BeginExpansion
\end{frame}%
%EndExpansion

\bigskip

%TCIMACRO{\TeXButton{BeginFrame}{\begin{frame}}}%
%BeginExpansion
\begin{frame}%
%EndExpansion

\QTR{frametitle}{Consumption and Labor Supply with Partial Insurance}

\QTR{framesubtitle}{Results (continued...)}

\begin{center}
\FRAME{ftbpF}{2.7172in}{2.5382in}{0pt}{}{}{Figure}{\special{language
"Scientific Word";type "GRAPHIC";maintain-aspect-ratio TRUE;display
"USEDEF";valid_file "T";width 2.7172in;height 2.5382in;depth
0pt;original-width 8.5988in;original-height 8.0298in;cropleft "0";croptop
"1";cropright "1";cropbottom "0";tempfilename
'MH3F4W00.wmf';tempfile-properties "XPR";}}
\end{center}

%TCIMACRO{\TeXButton{Transition: Box Out}{\transboxout}}%
%BeginExpansion
\transboxout%
%EndExpansion
%TCIMACRO{\TeXButton{EndFrame}{\end{frame}}}%
%BeginExpansion
\end{frame}%
%EndExpansion

\bigskip

%TCIMACRO{\TeXButton{BeginFrame}{\begin{frame}}}%
%BeginExpansion
\begin{frame}%
%EndExpansion

\QTR{frametitle}{Consumption and Labor Supply with Partial Insurance}

\QTR{framesubtitle}{Results (continued...)}

\begin{center}
\FRAME{dtbpF}{2.6948in}{2.6723in}{0pt}{}{}{Figure}{\special{language
"Scientific Word";type "GRAPHIC";maintain-aspect-ratio TRUE;display
"USEDEF";valid_file "T";width 2.6948in;height 2.6723in;depth
0pt;original-width 8.2659in;original-height 8.1958in;cropleft "0";croptop
"1";cropright "1";cropbottom "0";tempfilename
'MH3F4W01.wmf';tempfile-properties "XPR";}}
\end{center}

%TCIMACRO{\TeXButton{Transition: Box Out}{\transboxout}}%
%BeginExpansion
\transboxout%
%EndExpansion
%TCIMACRO{\TeXButton{EndFrame}{\end{frame}}}%
%BeginExpansion
\end{frame}%
%EndExpansion

\bigskip

%TCIMACRO{\TeXButton{BeginFrame}{\begin{frame}}}%
%BeginExpansion
\begin{frame}%
%EndExpansion

\QTR{frametitle}{Consumption and Labor Supply with Partial Insurance}

\QTR{framesubtitle}{Results (continued...)}

\begin{itemize}
\item The US and the model-simulated data have the following features:

\begin{enumerate}
\item The variance of log wages increases by around 35 log points,
approximately linearly, between ages $35-57$.

\item The variance of log consumption grows much less, 10 log points over
the life cycle: a significant share ($38\%$) of permanent shocks are
insurable.
\end{enumerate}

\item Figure 3 depicts the evolution of insurable and uninsurable wage
dispersions over time.
\end{itemize}

%TCIMACRO{\TeXButton{Transition: Box Out}{\transboxout}}%
%BeginExpansion
\transboxout%
%EndExpansion
%TCIMACRO{\TeXButton{EndFrame}{\end{frame}}}%
%BeginExpansion
\end{frame}%
%EndExpansion

\bigskip

%TCIMACRO{\TeXButton{BeginFrame}{\begin{frame}}}%
%BeginExpansion
\begin{frame}%
%EndExpansion

\QTR{frametitle}{Consumption and Labor Supply with Partial Insurance}

\QTR{framesubtitle}{Results (continued...)}

\begin{center}
\FRAME{dtbpF}{2.9931in}{2.8781in}{0pt}{}{}{Figure}{\special{language
"Scientific Word";type "GRAPHIC";maintain-aspect-ratio TRUE;display
"USEDEF";valid_file "T";width 2.9931in;height 2.8781in;depth
0pt;original-width 8.2659in;original-height 7.9459in;cropleft "0";croptop
"1";cropright "1";cropbottom "0";tempfilename
'MH3F4W02.wmf';tempfile-properties "XPR";}}
\end{center}

%TCIMACRO{\TeXButton{Transition: Box Out}{\transboxout}}%
%BeginExpansion
\transboxout%
%EndExpansion
%TCIMACRO{\TeXButton{EndFrame}{\end{frame}}}%
%BeginExpansion
\end{frame}%
%EndExpansion

\bigskip

%TCIMACRO{\TeXButton{BeginFrame}{\begin{frame}}}%
%BeginExpansion
\begin{frame}%
%EndExpansion

\QTR{frametitle}{Consumption and Labor Supply with Partial Insurance}

\QTR{framesubtitle}{Results (continued...)}

\begin{center}
\textbf{\FRAME{dtbpF}{4.0361in}{1.5117in}{0pt}{}{}{Figure}{\special{language
"Scientific Word";type "GRAPHIC";maintain-aspect-ratio TRUE;display
"USEDEF";valid_file "T";width 4.0361in;height 1.5117in;depth
0pt;original-width 13.3778in;original-height 4.9874in;cropleft "0";croptop
"1";cropright "1";cropbottom "0";tempfilename
'MHEBZD01.wmf';tempfile-properties "XPR";}}}
\end{center}

%TCIMACRO{\TeXButton{Transition: Box Out}{\transboxout}}%
%BeginExpansion
\transboxout%
%EndExpansion
%TCIMACRO{\TeXButton{EndFrame}{\end{frame}}}%
%BeginExpansion
\end{frame}%
%EndExpansion

\bigskip

%TCIMACRO{\TeXButton{BeginFrame}{\begin{frame}}}%
%BeginExpansion
\begin{frame}%
%EndExpansion

\QTR{frametitle}{Consumption and Labor Supply with Partial Insurance}

\QTR{framesubtitle}{Results (continued...)}

\begin{center}
\FRAME{ftbpF}{3.2136in}{1.3327in}{0pt}{}{}{Figure}{\special{language
"Scientific Word";type "GRAPHIC";maintain-aspect-ratio TRUE;display
"USEDEF";valid_file "T";width 3.2136in;height 1.3327in;depth
0pt;original-width 9.4879in;original-height 3.9176in;cropleft "0";croptop
"1";cropright "1";cropbottom "0";tempfilename
'MH3F4W03.wmf';tempfile-properties "XPR";}}
\end{center}

%TCIMACRO{\TeXButton{Transition: Box Out}{\transboxout}}%
%BeginExpansion
\transboxout%
%EndExpansion
%TCIMACRO{\TeXButton{EndFrame}{\end{frame}}}%
%BeginExpansion
\end{frame}%
%EndExpansion

\bigskip

%TCIMACRO{\TeXButton{BeginFrame}{\begin{frame}}}%
%BeginExpansion
\begin{frame}%
%EndExpansion

\QTR{frametitle}{Consumption and Labor Supply with Partial Insurance}

\QTR{framesubtitle}{Given these results... what's next?}

\begin{itemize}
\item Incorporate to the discussion simultaneously:

\begin{enumerate}
\item Labor decisions at the household level: give rise to insurance via
family labor supply.

\item External sources of insurance.

\item Self insurance via asset accumulation.
\end{enumerate}
\end{itemize}

%TCIMACRO{\TeXButton{Transition: Box Out}{\transboxout}}%
%BeginExpansion
\transboxout%
%EndExpansion
%TCIMACRO{\TeXButton{EndFrame}{\end{frame}}}%
%BeginExpansion
\end{frame}%
%EndExpansion

\bigskip 

%TCIMACRO{\TeXButton{BeginFrame}{\begin{frame}}}%
%BeginExpansion
\begin{frame}%
%EndExpansion

\QTR{frametitle}{Consumption Inequality and Family Labor Supply}

\bigskip 

\begin{center}
\textbf{Consumption Inequality and Family Labor Supply}

\textit{Richard Blundell}

\textit{Luigi Pistaferri}

\textit{Itay Saporta-Eksten}
\end{center}

\bigskip 

%TCIMACRO{\TeXButton{Transition: Box Out}{\transboxout}}%
%BeginExpansion
\transboxout%
%EndExpansion
%TCIMACRO{\TeXButton{EndFrame}{\end{frame}}}%
%BeginExpansion
\end{frame}%
%EndExpansion

\section{Consumption Inequality and Family Labor Supply}

%TCIMACRO{\TeXButton{BeginFrame}{\begin{frame}}}%
%BeginExpansion
\begin{frame}%
%EndExpansion

\QTR{frametitle}{Consumption Inequality and Family Labor Supply}

\QTR{framesubtitle}{Model}

\begin{itemize}
\item Set up a model that allows for three potential sources of smoothing:

\begin{enumerate}
\item Self-insurance through credit markets.

\item Family labor supply (hours of work can be adjusted along with, or
alternatively to, spending on goods in response to shocks to economic
resources).

\item External sources of insurance (from help received by networks of
relatives and friends to social insurance such as UB).
\end{enumerate}

\item Framework:
\end{itemize}

\begin{enumerate}
\item Life-cycle setup in which two individuals (husband \& wife) make
unitary decisions about household consumption and their individual labor
supply.

\item There is uncertainty about offered market wages.

\item Allow for partial insurance through assets accumulation; heterogeneous
Frisch elasticities; non-separability of consumption and leisure;
differences between extensive and intensive margins of labor supply.
\end{enumerate}

%TCIMACRO{\TeXButton{Transition: Box Out}{\transboxout}}%
%BeginExpansion
\transboxout%
%EndExpansion
%TCIMACRO{\TeXButton{EndFrame}{\end{frame}}}%
%BeginExpansion
\end{frame}%
%EndExpansion

\bigskip

%TCIMACRO{\TeXButton{BeginFrame}{\begin{frame}}}%
%BeginExpansion
\begin{frame}%
%EndExpansion

\QTR{frametitle}{Consumption Inequality and Family Labor Supply}

\QTR{framesubtitle}{Wage Process}

\begin{itemize}
\item Permanent-transitory type wage process for each earner (Permanent
component evolves as a unit root process).

\item The log real wage of individual $j=1,2$ in household $i=1,\ldots ,N$
at time $t=1,\ldots ,T$ is%
\begin{eqnarray}
\log W_{i,j,t} &=&x_{i,j,t}^{\prime }\beta _{W}^{j}+F_{i,j,t}+u_{i,j,t} \\
F_{i,j,t} &=&F_{i,j,t-1}+v_{i,j,t}  \notag
\end{eqnarray}%
where $x_{i,j,t}^{\prime }$ are characteristics that influence wages and are
known to the household; $u_{i,j,t}$ and $v_{i,j,t}$ are the transitory and
permanent shocks, respectively.

\item This process adequately fits the data, although it is far from
controversial (superior information and heterogeneous growth issues are
studied in related literature).
\end{itemize}

%TCIMACRO{\TeXButton{Transition: Box Out}{\transboxout}}%
%BeginExpansion
\transboxout%
%EndExpansion
%TCIMACRO{\TeXButton{EndFrame}{\end{frame}}}%
%BeginExpansion
\end{frame}%
%EndExpansion

\bigskip

%TCIMACRO{\TeXButton{BeginFrame}{\begin{frame}}}%
%BeginExpansion
\begin{frame}%
%EndExpansion

\QTR{frametitle}{Consumption Inequality and Family Labor Supply}

\QTR{framesubtitle}{Wage Process (continued...)}

\begin{itemize}
\item The "across-time" and "across-shocks" moments are given by%
\begin{eqnarray}
\mathbb{E}\left( u_{i,j,t},u_{i,j,t-s}\right) &=&\left\{ 
\begin{array}{cc}
\sigma _{u_{j}}^{2} & j=k,s=0 \\ 
\sigma _{u_{j}u_{k}} & j\neq k,s=0 \\ 
0 & otherwise%
\end{array}%
\right. \\
\mathbb{E}\left( v_{i,j,t},v_{i,j,t-s}\right) &=&\left\{ 
\begin{array}{cc}
\sigma _{v_{j}}^{2} & j=k,s=0 \\ 
\sigma _{v_{j}v_{k}} & j\neq k,s=0 \\ 
0 & otherwise%
\end{array}%
\right.  \notag
\end{eqnarray}

\item The process is time-invariant, serially uncorrelated, correlated
across spouses (direction is theoretically ambiguous: assortative matching
vs. negatively correlated job selection).
\end{itemize}

%TCIMACRO{\TeXButton{Transition: Box Out}{\transboxout}}%
%BeginExpansion
\transboxout%
%EndExpansion
%TCIMACRO{\TeXButton{EndFrame}{\end{frame}}}%
%BeginExpansion
\end{frame}%
%EndExpansion

\bigskip

%TCIMACRO{\TeXButton{BeginFrame}{\begin{frame}}}%
%BeginExpansion
\begin{frame}%
%EndExpansion

\QTR{frametitle}{Consumption Inequality and Family Labor Supply}

\QTR{framesubtitle}{Household Problem}

\begin{equation}
\max_{\left\{ C_{i,t+s},H_{i,1,t+s},H_{i,2,t+s}\right\} _{s=0}^{T-t}}\mathbb{%
E}_t\dsum\limits_{s=0}^{T-t}u_{t+s}\left( 
\begin{array}{c}
C_{i,t+s},H_{i,1,t+s},H_{i,2,t+s}; \\ 
z_{i,t+s},z_{i,1,t+s},z_{i,2,t+s}%
\end{array}%
\right)
\end{equation}%
s.t.%
\begin{equation}
A_{i,t+1}=\left( 1+r\right) \left(
A_{i,t}+H_{i,1,t}W_{i,1,t}+H_{i,2,t}W_{i,2,t}-C_{i,t}\right) .
\end{equation}

\begin{itemize}
\item $z_{i,t+s},z_{i,1,t+s},z_{i,2,t+s}$ are per-period utility shifters
which are household and spouse specific, respectively.

\item This general setup enables to consider the additive and non-additive
separabilities.
\end{itemize}

%TCIMACRO{\TeXButton{Transition: Box Out}{\transboxout}}%
%BeginExpansion
\transboxout%
%EndExpansion
%TCIMACRO{\TeXButton{EndFrame}{\end{frame}}}%
%BeginExpansion
\end{frame}%
%EndExpansion

\bigskip

%TCIMACRO{\TeXButton{BeginFrame}{\begin{frame}}}%
%BeginExpansion
\begin{frame}%
%EndExpansion

\QTR{frametitle}{Consumption Inequality and Family Labor Supply}

\QTR{framesubtitle}{Analytical Solution}

\begin{itemize}
\item Not found in general under this specification (only under strong
assumptions about utility). Solution: use approximation methods.

\item Use Taylor approximation to the log-linearized forms of the f.o.c.s
and the intertemporal budget constraint.

\item Found this for the cases of additive separability, non-additive
separability, measurement error, and non-linear taxation.
\end{itemize}

%TCIMACRO{\TeXButton{Transition: Box Out}{\transboxout}}%
%BeginExpansion
\transboxout%
%EndExpansion
%TCIMACRO{\TeXButton{EndFrame}{\end{frame}}}%
%BeginExpansion
\end{frame}%
%EndExpansion

\bigskip

%TCIMACRO{\TeXButton{BeginFrame}{\begin{frame}}}%
%BeginExpansion
\begin{frame}%
%EndExpansion

\QTR{frametitle}{Consumption Inequality and Family Labor Supply}

\QTR{framesubtitle}{Additive Separability: Approximation to the Euler
Equations and the BC}

The system that described the solution in log difference of the relevant
variables is

\begin{equation}
\left[ 
\begin{array}{c}
\Delta c_{i,t} \\ 
\Delta y_{i,1,t} \\ 
\Delta y_{i,2,t}%
\end{array}%
\right] =\left[ 
\begin{array}{cccc}
0 & 0 & \varkappa _{c,v_{1}} & \varkappa _{c,v_{2}} \\ 
\varkappa _{y_{1},u_{1}} & 0 & \varkappa _{y_{1},v_{1}} & \varkappa
_{y_{1},v_{2}} \\ 
0 & \varkappa _{y_{2},u_{2}} & \varkappa _{y_{2},v_{1}} & \varkappa
_{y_{2},v_{2}}%
\end{array}%
\right] \left[ 
\begin{array}{c}
\Delta u_{i,1,t} \\ 
\Delta u_{i,2,t} \\ 
v_{i,1,t} \\ 
v_{i,2,t}%
\end{array}%
\right]
\end{equation}

%TCIMACRO{\TeXButton{Transition: Box Out}{\transboxout}}%
%BeginExpansion
\transboxout%
%EndExpansion
%TCIMACRO{\TeXButton{EndFrame}{\end{frame}}}%
%BeginExpansion
\end{frame}%
%EndExpansion

\bigskip

%TCIMACRO{\TeXButton{BeginFrame}{\begin{frame}}}%
%BeginExpansion
\begin{frame}%
%EndExpansion

\QTR{frametitle}{Consumption Inequality and Family Labor Supply}

\QTR{framesubtitle}{Additive Separability: Approximation to the Euler
Equations and the BC (continued...)}

where

\begin{eqnarray}
\varkappa _{c,v_{j}} &=&-\frac{\eta _{c,p}\left( 1-\pi _{i,t}\right)
s_{i,j,y}\left( 1-\eta _{h_{j},w_{j}}\right) }{\eta _{c,p}+\left( 1-\pi
_{i,t}\right) \overset{\_}{\eta _{h,w}}} \\
\varkappa _{y_{j},u_{j}} &=&1+\eta _{h_{j},w_{j}}  \notag \\
\varkappa _{y_{j},v_{j}} &=&1+\eta _{h_{j},w_{j}}\left( 1-\frac{\left( 1-\pi
_{i,t}\right) s_{i,j,y}\left( 1-\eta _{h_{j},w_{j}}\right) }{\eta
_{c,p}+\eta _{c,p}+\left( 1-\pi _{i,t}\right) \overset{\_}{\eta _{h,w}}}%
\right)  \notag \\
\varkappa _{y_{j},v_{-j}} &=&-\frac{\eta _{h_{j},w_{j}}\left( 1-\pi
_{i,t}\right) s_{i,-j,y}\left( 1-\eta _{h_{-j},w_{-j}}\right) }{\eta
_{c,p}+\left( 1-\pi _{i,t}\right) \overset{\_}{\eta _{h,w}}}  \notag
\end{eqnarray}

%TCIMACRO{\TeXButton{Transition: Box Out}{\transboxout}}%
%BeginExpansion
\transboxout%
%EndExpansion
%TCIMACRO{\TeXButton{EndFrame}{\end{frame}}}%
%BeginExpansion
\end{frame}%
%EndExpansion

\bigskip

%TCIMACRO{\TeXButton{BeginFrame}{\begin{frame}}}%
%BeginExpansion
\begin{frame}%
%EndExpansion

\QTR{frametitle}{Consumption Inequality and Family Labor Supply}

\QTR{framesubtitle}{Additive Separability: Approximation to the Euler
Equations and the BC (continued...)}

and 
\begin{eqnarray*}
\eta _{c,p} &=&-\frac{u_{C}}{u_{CC}}\frac{1}{c} \\
&>&0 \\
\eta _{h_{j},w_{j}} &=&\frac{g_{H_{j}}^{j}}{g_{H_{j}H_{j}}^{j}}\frac{1}{H_{j}%
} \\
&>&0
\end{eqnarray*}%
are the EIS for consumption and labor supply of earner $j$, respectively.

%TCIMACRO{\TeXButton{Transition: Box Out}{\transboxout}}%
%BeginExpansion
\transboxout%
%EndExpansion
%TCIMACRO{\TeXButton{EndFrame}{\end{frame}}}%
%BeginExpansion
\end{frame}%
%EndExpansion

\bigskip

%TCIMACRO{\TeXButton{BeginFrame}{\begin{frame}}}%
%BeginExpansion
\begin{frame}%
%EndExpansion

\QTR{frametitle}{Consumption Inequality and Family Labor Supply}

\QTR{framesubtitle}{Additive Separability: Approximation to the Euler
Equations and the BC (continued...)}

and 
\begin{eqnarray*}
\pi _{i,t} &=&\frac{Assets_{i,t}}{Assets_{i,t}+HumanWealth_{i,t}} \\
s_{i,j,y} &=&\frac{HumanWealth_{i,j,t}}{HumanWealth_{i,t}} \\
\overset{\_}{\eta _{h,w}} &=&\dsum\limits_{j=1}^{2}s_{i,j,t}\eta
_{h_{j},w_{j}}
\end{eqnarray*}%
are the partial insurance coefficient, the share of earner j's human wealth
over family human wealth, and the household's weighted average of the EIS of
labor supply of the two earners.

%TCIMACRO{\TeXButton{Transition: Box Out}{\transboxout}}%
%BeginExpansion
\transboxout%
%EndExpansion
%TCIMACRO{\TeXButton{EndFrame}{\end{frame}}}%
%BeginExpansion
\end{frame}%
%EndExpansion

\bigskip

%TCIMACRO{\TeXButton{BeginFrame}{\begin{frame}}}%
%BeginExpansion
\begin{frame}%
%EndExpansion

\QTR{frametitle}{Consumption Inequality and Family Labor Supply}

\QTR{framesubtitle}{Insurance above Self-Insurance }

\begin{itemize}
\item Households have access to multiple external sources of insurance.

\item This is hard to credibly model: myriad of external insurance channel.

\item Subsume this mechanism into one parameter, $\beta $, that captures the
proportion if this insurance.

\item The share $\pi _{i,t}$ is multiplied by $\left( 1-\beta \right) $ when
it appears.

\item The parameter measures all consumption insurance that remains after
accounting for the "self insurance" represented by asset accumulation and
labor supply.

\item $\beta >0$: external insurance; $\beta =0$ no external insurance; $%
\beta <0$ (consumption over-responds to shocks given illiquid forms of asset
accumulation/transaction costs exceeding benefits from smoothing).
\end{itemize}

%TCIMACRO{\TeXButton{Transition: Box Out}{\transboxout}}%
%BeginExpansion
\transboxout%
%EndExpansion
%TCIMACRO{\TeXButton{EndFrame}{\end{frame}}}%
%BeginExpansion
\end{frame}%
%EndExpansion

\bigskip 

%TCIMACRO{\TeXButton{BeginFrame}{\begin{frame}}}%
%BeginExpansion
\begin{frame}%
%EndExpansion

\QTR{frametitle}{Consumption Inequality and Family Labor Supply}

\QTR{framesubtitle}{Identification, Data, Inference, and Estimation Issues}

\begin{itemize}
\item Before discussing identification consider the following summary of
parameters

\begin{itemize}
\item $\left( \sigma _{\upsilon _{1}}^{2},\sigma _{\upsilon _{2}}^{2},\sigma
_{u_{1}}^{2},\sigma _{u_{2}}^{2},\sigma _{\upsilon _{1}\upsilon _{2}},\sigma
_{u_{1}u_{2}}\right) ,$ variances of the permanent and transitory components
of the shocks and respective within household covariances (stationary). 

\item  $\left( \pi ,s,\beta \right) ,$insurance parameters. 

\item $\left( \eta _{h_{1}w_{1},}\eta _{h_{2}w_{2},}\eta _{cp,}\eta
_{h_{1}w_{2},}\eta _{h_{2}w_{1},}\eta _{cw_{1},}\eta _{cw_{2},}\eta
_{h_{1}p,}\eta _{h_{2}p}\right) ,$ direct and crossed Frisch elasticities of
consumptions and hours worked by each individual in the household. 
\end{itemize}
\end{itemize}

%TCIMACRO{\TeXButton{Transition: Box Out}{\transboxout}}%
%BeginExpansion
\transboxout%
%EndExpansion
%TCIMACRO{\TeXButton{EndFrame}{\end{frame}}}%
%BeginExpansion
\end{frame}%
%EndExpansion

\bigskip 

%TCIMACRO{\TeXButton{BeginFrame}{\begin{frame}}}%
%BeginExpansion
\begin{frame}%
%EndExpansion

\QTR{frametitle}{Consumption Inequality and Family Labor Supply}

\QTR{framesubtitle}{Identification, Data, Inference, and Estimation Issues}

\begin{itemize}
\item Identification of the wage parameters is obtained through a strategy
followed by Meghir and Pistaferri (2004) that is based on the equation that
represents the wage growth of earner $j$ at time $t$, $\Delta w_{j,t}=\Delta
u_{j,t}+\upsilon _{j,t}$:
\end{itemize}

\begin{eqnarray}
\sigma _{u_{j}}^{2} &=&-\mathbb{E}\left( \Delta w_{j,t},\Delta
w_{j,t+1}\right) \\
\sigma _{\upsilon _{j}}^{2} &=&\mathbb{E}\left( \Delta w_{j,t}\left( \Delta
w_{j,t+1}+\Delta w_{j,t}-\Delta w_{j,t-1}\right) \right)  \notag \\
\sigma _{u_{1}u_{2}} &=&-\mathbb{E}\left( \Delta w_{1,t},\Delta
w_{2,t+1}\right)  \notag \\
\sigma _{\upsilon _{1}\upsilon _{2}} &=&\mathbb{E}\left( \Delta
w_{1,t}\left( \Delta w_{2,t+1}+\Delta w_{2,t}-\Delta w_{2,t-1}\right) \right)
\notag
\end{eqnarray}

%TCIMACRO{\TeXButton{Transition: Box Out}{\transboxout}}%
%BeginExpansion
\transboxout%
%EndExpansion
%TCIMACRO{\TeXButton{EndFrame}{\end{frame}}}%
%BeginExpansion
\end{frame}%
%EndExpansion

\bigskip

%TCIMACRO{\TeXButton{BeginFrame}{\begin{frame}}}%
%BeginExpansion
\begin{frame}%
%EndExpansion

\QTR{frametitle}{Consumption Inequality and Family Labor Supply}

\QTR{framesubtitle}{Identification, Data, Inference, and Estimation Issues
(continued...)}

\begin{itemize}
\item Identification of the rest of the parameters comes from the cross
moments of the wage growth, the earnings growth, and the use of the symmetry
of the Frisch substitution matrix.

\item Identification of $\pi ,s$ comes strictly from the data.

\item The authors use the PSID data for the periods 1999-2009.

\item The authors generalize the model above to allow for measurement error
in consumption, wages, and earnings and identification is still possible.

\item Roughly speaking, the estimation is based on the use of multiple
moments by the method of GMM (with an identity matrix as the weighting
scheme)

\item Inference is based on block bootstrap (i.e., re-samples are taken at
the individual level).
\end{itemize}

%TCIMACRO{\TeXButton{Transition: Box Out}{\transboxout}}%
%BeginExpansion
\transboxout%
%EndExpansion
%TCIMACRO{\TeXButton{EndFrame}{\end{frame}}}%
%BeginExpansion
\end{frame}%
%EndExpansion

\bigskip

%TCIMACRO{\TeXButton{BeginFrame}{\begin{frame}}}%
%BeginExpansion
\begin{frame}%
%EndExpansion

\QTR{frametitle}{Consumption Inequality and Family Labor Supply}

\QTR{framesubtitle}{Identification, Data, Inference, and Estimation Issues
(continued...)}

\begin{itemize}
\item The model induces interior solutions for labor supply for both spouses.

\item A major concern when modeling labor supply is endogenous selection
into work: need to distinguish between extensive and intensive margins.

\item 93\% of male is employed in the data: justification for using only
employed male.

\item 80\% of the wives of this 93\% of males is employed, which makes
accounting for selection potentially important.

\item Follow Low et al. (2010). Use state year dummies intended to capture
labor market related policy changed at the state level and the presence of
first and second mortgages (some evidence shows that female participation
rises when households move into home ownership).
\end{itemize}

%TCIMACRO{\TeXButton{Transition: Box Out}{\transboxout}}%
%BeginExpansion
\transboxout%
%EndExpansion
%TCIMACRO{\TeXButton{EndFrame}{\end{frame}}}%
%BeginExpansion
\end{frame}%
%EndExpansion

\bigskip

%TCIMACRO{\TeXButton{BeginFrame}{\begin{frame}}}%
%BeginExpansion
\begin{frame}%
%EndExpansion

\QTR{frametitle}{Consumption Inequality and Family Labor Supply}

\QTR{framesubtitle}{Empirical Strategy }

\begin{enumerate}
\item Regress log differences of the relevant variables (consumption, wages
and earnings) onto observable characteristics and construct first difference
residuals for each of them.

\item Estimate the wage variances and covariances using the second moments
of the log first differences.

\item Estimate the smoothing parameters using asset and (current and
projected) earnings data.

\item Estimate the preference parameters using the second moments for
earnings and consumption conditioning in the results obtained in the
previous steps.
\end{enumerate}

%TCIMACRO{\TeXButton{Transition: Box Out}{\transboxout}}%
%BeginExpansion
\transboxout%
%EndExpansion
%TCIMACRO{\TeXButton{EndFrame}{\end{frame}}}%
%BeginExpansion
\end{frame}%
%EndExpansion

\bigskip

%TCIMACRO{\TeXButton{BeginFrame}{\begin{frame}}}%
%BeginExpansion
\begin{frame}%
%EndExpansion

\QTR{frametitle}{Consumption Inequality and Family Labor Supply}

\QTR{framesubtitle}{Results: Consumption and Labor Supply Parameters}

\begin{center}
\textbf{\FRAME{dtbpF}{4.4192in}{2.4777in}{0pt}{}{}{Figure}{\special{language
"Scientific Word";type "GRAPHIC";maintain-aspect-ratio TRUE;display
"USEDEF";valid_file "T";width 4.4192in;height 2.4777in;depth
0pt;original-width 14.9898in;original-height 8.3766in;cropleft "0";croptop
"1";cropright "1";cropbottom "0";tempfilename
'MHEC1D02.wmf';tempfile-properties "XPR";}}}
\end{center}

%TCIMACRO{\TeXButton{Transition: Box Out}{\transboxout}}%
%BeginExpansion
\transboxout%
%EndExpansion
%TCIMACRO{\TeXButton{EndFrame}{\end{frame}}}%
%BeginExpansion
\end{frame}%
%EndExpansion

\bigskip

%TCIMACRO{\TeXButton{BeginFrame}{\begin{frame}}}%
%BeginExpansion
\begin{frame}%
%EndExpansion

\QTR{frametitle}{Consumption Inequality and Family Labor Supply}

\QTR{framesubtitle}{Results: Smoothing Parameters \FRAME{dtbpF}{3.5777in}{%
2.9836in}{0pt}{}{}{Figure}{\special{language "Scientific Word";type
"GRAPHIC";maintain-aspect-ratio TRUE;display "USEDEF";valid_file "T";width
3.5777in;height 2.9836in;depth 0pt;original-width 9.3772in;original-height
7.8075in;cropleft "0";croptop "1";cropright "1";cropbottom "0";tempfilename
'MH3F4W04.wmf';tempfile-properties "XPR";}}}

%TCIMACRO{\TeXButton{Transition: Box Out}{\transboxout}}%
%BeginExpansion
\transboxout%
%EndExpansion
%TCIMACRO{\TeXButton{EndFrame}{\end{frame}}}%
%BeginExpansion
\end{frame}%
%EndExpansion

\bigskip

%TCIMACRO{\TeXButton{BeginFrame}{\begin{frame}}}%
%BeginExpansion
\begin{frame}%
%EndExpansion

\QTR{frametitle}{Consumption Inequality and Family Labor Supply}

\QTR{framesubtitle}{Results: Smoothing Parameters (continued)}

\begin{center}
\FRAME{dtbpF}{3.3096in}{2.7095in}{0pt}{}{}{Figure}{\special{language
"Scientific Word";type "GRAPHIC";maintain-aspect-ratio TRUE;display
"USEDEF";valid_file "T";width 3.3096in;height 2.7095in;depth
0pt;original-width 8.8211in;original-height 7.2099in;cropleft "0";croptop
"1";cropright "1";cropbottom "0";tempfilename
'MH3F4W05.wmf';tempfile-properties "XPR";}}
\end{center}

%TCIMACRO{\TeXButton{Transition: Box Out}{\transboxout}}%
%BeginExpansion
\transboxout%
%EndExpansion
%TCIMACRO{\TeXButton{EndFrame}{\end{frame}}}%
%BeginExpansion
\end{frame}%
%EndExpansion

\bigskip

%TCIMACRO{\TeXButton{BeginFrame}{\begin{frame}}}%
%BeginExpansion
\begin{frame}%
%EndExpansion

\QTR{frametitle}{Consumption Inequality and Family Labor Supply}

\QTR{framesubtitle}{Results: Smoothing Parameters (continued)}

\begin{itemize}
\item Figure 1...
\end{itemize}

\begin{enumerate}
\item Self-insurance due to asset accumulation is negligible at the
beginning of the life-cycle.

\item Asset accumulation and the decline of human capital due to shortening
of time horizon imply and increase in $\pi _{i,t}$ as time goes by, and
hence households are able to smooth permanent wage shocks better overtime.
\end{enumerate}

\begin{itemize}
\item Figure 2...

\begin{enumerate}
\item Life-cycle evolution if the distribution of earnings power within the
household.

\item On average, the husband commands about $2/3$ of the total of the total
household human wealth.

\item The peak is, presumably, due to fertility decisions.
\end{enumerate}
\end{itemize}

%TCIMACRO{\TeXButton{Transition: Box Out}{\transboxout}}%
%BeginExpansion
\transboxout%
%EndExpansion
%TCIMACRO{\TeXButton{EndFrame}{\end{frame}}}%
%BeginExpansion
\end{frame}%
%EndExpansion

\bigskip

%TCIMACRO{\TeXButton{BeginFrame}{\begin{frame}}}%
%BeginExpansion
\begin{frame}%
%EndExpansion

\QTR{frametitle}{Consumption Inequality and Family Labor Supply}

\QTR{framesubtitle}{Results: Wage Variances}

\begin{center}
\FRAME{dtbpF}{2.501in}{2.674in}{0pt}{}{}{Figure}{\special{language
"Scientific Word";type "GRAPHIC";maintain-aspect-ratio TRUE;display
"USEDEF";valid_file "T";width 2.501in;height 2.674in;depth
0pt;original-width 7.5991in;original-height 8.1266in;cropleft "0";croptop
"1";cropright "1";cropbottom "0";tempfilename
'MH3F4W06.wmf';tempfile-properties "XPR";}}
\end{center}

%TCIMACRO{\TeXButton{Transition: Box Out}{\transboxout}}%
%BeginExpansion
\transboxout%
%EndExpansion
%TCIMACRO{\TeXButton{EndFrame}{\end{frame}}}%
%BeginExpansion
\end{frame}%
%EndExpansion

\bigskip

%TCIMACRO{\TeXButton{BeginFrame}{\begin{frame}}}%
%BeginExpansion
\begin{frame}%
%EndExpansion

\QTR{frametitle}{Consumption Inequality and Family Labor Supply}

\QTR{framesubtitle}{Results: Wage Variances (continued...)}

\begin{enumerate}
\item "Wage instability" defined as variance of the transitory component
(Gottschalk \& Moffitt, 2008) are larger for males (larger influence of
turnover, etc.).

\item The variance of the more structural component (permanent) is similar
for men and women (higher for women maybe due to greater dispersion in
returns to unobserved skills).

\item Both the transitory and permanent components are positively
correlated. This could evidence assortative matching.
\end{enumerate}

%TCIMACRO{\TeXButton{Transition: Box Out}{\transboxout}}%
%BeginExpansion
\transboxout%
%EndExpansion
%TCIMACRO{\TeXButton{EndFrame}{\end{frame}}}%
%BeginExpansion
\end{frame}%
%EndExpansion

\bigskip

%TCIMACRO{\TeXButton{BeginFrame}{\begin{frame}}}%
%BeginExpansion
\begin{frame}%
%EndExpansion

\QTR{frametitle}{Consumption Inequality and Family Labor Supply}

\QTR{framesubtitle}{Results: Consumption and Labor Supply Parameters
(continued...)}

\begin{itemize}
\item Separability case:

\begin{enumerate}
\item EIS is estimated $.2$, implying a relative risk aversion coefficient
of around $5$ (fairly high but in the ballpark of previous estimates in this
literature).

\item Frisch elasticity of men labor supply is estimated to be $.4$, which
ranges in in values given by MaCurdy (1981), Altonji's (1986), and Keane
(2011).

\item Frisch elasticity of women labor supply is estimated to be $.8$, in
comparison of the estimate of $1$ obtained by MaCurdy and Heckman (1980) and
consistent with high values found in this literature surveyed by Keane
(2011).

\item Implausible value for $\beta =.74$, which implies a very large amount
of "external" insurance over and above self-insurance. This value implies
and excessive degree of consumption smoothing.
\end{enumerate}
\end{itemize}

%TCIMACRO{\TeXButton{Transition: Box Out}{\transboxout}}%
%BeginExpansion
\transboxout%
%EndExpansion
%TCIMACRO{\TeXButton{EndFrame}{\end{frame}}}%
%BeginExpansion
\end{frame}%
%EndExpansion

\bigskip

%TCIMACRO{\TeXButton{BeginFrame}{\begin{frame}}}%
%BeginExpansion
\begin{frame}%
%EndExpansion

\QTR{frametitle}{Consumption Inequality and Family Labor Supply}

\QTR{framesubtitle}{Results: Consumption and Labor Supply Parameters
(continued...)}

\begin{itemize}
\item Non-Separability

\begin{enumerate}
\item Additive separability is strongly rejected: four individual parameters
particular to this model are significantly different than zero. 

\item Complementarity of husband and wife leisure.

\item Both leisures (husband and wife) are complements with consumption.

\item $\beta $ is not significant in this case, which gives the author
another reason to have this model as their preferred.
\end{enumerate}
\end{itemize}

%TCIMACRO{\TeXButton{Transition: Box Out}{\transboxout}}%
%BeginExpansion
\transboxout%
%EndExpansion
%TCIMACRO{\TeXButton{EndFrame}{\end{frame}}}%
%BeginExpansion
\end{frame}%
%EndExpansion

\bigskip

%TCIMACRO{\TeXButton{BeginFrame}{\begin{frame}}}%
%BeginExpansion
\begin{frame}%
%EndExpansion

\QTR{frametitle}{Consumption Inequality and Family Labor Supply}

\QTR{framesubtitle}{Discussion: Advance Information, Extensive Margin vs.
Intensive Margin, Non-Linear Taxes}

\begin{itemize}
\item The mute response to consumption to wage shocks may be due to wage
changes not being shocks.

\item The authors follow Cuhna et al. (2005) and test of future wage
predicts current consumption growth to test for advance "information".

\begin{itemize}
\item They compute $\mathbb{E}\left( \Delta c_{i,t},\Delta w_{i,j,t+\tau
}\right) $ for and test their joint significance. They cannot reject the
null of zero correlation.
\end{itemize}

\item Evidence for complementarity between leisure and consumption has been
rare. Authors like Aguiar and Hurst (2005) find substitution among this
goods.
\end{itemize}

%TCIMACRO{\TeXButton{Transition: Box Out}{\transboxout}}%
%BeginExpansion
\transboxout%
%EndExpansion
%TCIMACRO{\TeXButton{EndFrame}{\end{frame}}}%
%BeginExpansion
\end{frame}%
%EndExpansion

\bigskip

%TCIMACRO{\TeXButton{BeginFrame}{\begin{frame}}}%
%BeginExpansion
\begin{frame}%
%EndExpansion

\QTR{frametitle}{Consumption Inequality and Family Labor Supply}

\QTR{framesubtitle}{Discussion: Advance Information, Extensive Margin vs.
Intensive Margin, Non-Linear Taxes (continued...)}

\begin{itemize}
\item However, the "usual" evidence comes from studying the relationship
between changes in consumption and large changes in hours, often associated
with unemployment, retirement, etc. (extensive margin).

\item This paper focuses in the relationship between changes in consumption
and small changes in hours (vastly employed sub-sample).

\item The authors also extend the model to include non-linear progressive
taxation. This does not affect qualitatively the results and has "small"
impacts (Frisch elasticities are typically larger due to the feedback effect
of taxes).
\end{itemize}

%TCIMACRO{\TeXButton{Transition: Box Out}{\transboxout}}%
%BeginExpansion
\transboxout%
%EndExpansion
%TCIMACRO{\TeXButton{EndFrame}{\end{frame}}}%
%BeginExpansion
\end{frame}%
%EndExpansion

\bigskip

%TCIMACRO{\TeXButton{BeginFrame}{\begin{frame}}}%
%BeginExpansion
\begin{frame}%
%EndExpansion

\QTR{frametitle}{Consumption Inequality and Family Labor Supply}

\QTR{framesubtitle}{Discussion: A Summary Exercise with the Estimated
Parameters}

\begin{center}
\FRAME{ftbpF}{3.4765in}{2.6991in}{0pt}{}{}{Figure}{\special{language
"Scientific Word";type "GRAPHIC";maintain-aspect-ratio TRUE;display
"USEDEF";valid_file "T";width 3.4765in;height 2.6991in;depth
0pt;original-width 10.933in;original-height 8.4752in;cropleft "0";croptop
"1";cropright "1";cropbottom "0";tempfilename
'MH3F4W07.wmf';tempfile-properties "XPR";}}
\end{center}

%TCIMACRO{\TeXButton{Transition: Box Out}{\transboxout}}%
%BeginExpansion
\transboxout%
%EndExpansion
%TCIMACRO{\TeXButton{EndFrame}{\end{frame}}}%
%BeginExpansion
\end{frame}%
%EndExpansion

\end{document}
