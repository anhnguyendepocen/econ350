\documentclass[12pt]{article}

%Input preamble, math commands, environments, etc. from a saingle file.
\documentclass[12pt]{article}

%Input preamble, math commands, environments, etc. from a saingle file. 
\documentclass[12pt]{article}

%Input preamble, math commands, environments, etc. from a saingle file. 
\documentclass[12pt]{article}

%Input preamble, math commands, environments, etc. from a saingle file. 
\input{HelperFiles/CourseDescription}

\begin{document}

\title{\textbf{Course Description Proposal: Econ 350, The Origins \& Consequences of Inequality in Capabilities }}
\author{Prepared by: Jorge L. Garc\'{i}a and Yike Wang \\ The University of Chicago}
\date{This Draft: \today}
\maketitle

\begin{abstract}
\noindent This course analyzes the origins and consequences of inequality in a life-cycle perspective. It uses Economic Models and Statistical Frameworks to study nine sub-topics that lead to a better understanding of a series of facts about the development of current socio-economic inequality, as described in Topic 1. In fact, the goal is to have, at the end of the course, an understanding of this phenomenon based on rigorous Economic and Econometric Methods.   
\end{abstract}

\section{General Topics}
\begin{enumerate}
\item The Dimensions of Inequality and Social Mobility
\begin{enumerate}
\item Basic Facts and Trends
\item Prices, Endowments, and Shocks
\item Skills
\item Transfers and Taxes
\item The Role of Family and Policy on Endowments
\item Philosophical \ldots
\end{enumerate}
\item Education and Inequality
\begin{enumerate}
\item Causal Inference and Instrumental Variables
\item Structural Models
\item Evidence
\item The Role of Capabilities as a Broader Notion of Skills
\end{enumerate}
\item Shocks and Uncertainty
\begin{enumerate}
\item Internal and External Variability
\item Modeling Uncertainty
\end{enumerate}
\item The Dynamics of Skill Formation
\begin{enumerate}
\item B.P. + Technology / Adult
\item The Creation of Initial Conditions 
\end{enumerate}
\item Factor Models and Models of Family Influence
\item Genes and Environments
\item Health: Life-Cycle Origins and Determinants
\item Social Mobility: Determinants and Models
\item Summary
\end{enumerate}

\section{Reading Lists}
\begin{enumerate}
\item The core reading list offers a guide to the literature in each of the class' topics and  is here: (\textbf{link to core reading list}).
\item The supplementary reading list offers students the possibility for further investigation of the class' topics and is here: (\textbf{link to supplementary reading list}).
\item The technical reading list provides background on the tools to do research on the frontier of the class' topics and is here: (\textbf{link to technical  reading list}).
\item Lecture handouts will be posted as available. Check the website and your email before class. The TAs will let you know when new handouts are available.
\item For some of the sections, there are videos from the 2012 and 2013 Summer School on Socio-Economics Inequality. These are useful as introductory summaries to the main themes of the course.
\end{enumerate}

\section{Lectures}
\noindent The class will meet in Rosenwald 301. Professor Heckman will teach on Tuesdays from 3:00 p.m. to 6:00 p.m. with a 15 minute break in between. The class will occasionally meet on Wednesdays in the same schedule. Students will need to read the relevant materials beforehand (the TAs will notify which are these materials in advance). Students will be expected to create a learning dynamics with relevant questions that stimulate discussions about the class topics. The TA session will meet in Rowsenwald 301 on Mondays from 3:00 p.m. to 6:00 p.m. The TA sessions will have two objectives: (i) cover topics not discussed in class due to time limitations; (ii) host a lecture on Structural Econometrics and Estimation by Philip Eisenhauer.

\section{Grading}
\subsection{Problem Sets}
\noindent There will be eight problem sets throughout the quarter. Students can form groups of three to answer the problem set and hand in one answer key per group. Although we cannot monitor this, we encourage students not to divide the questions and answer them individually. Group discussions improve the quality of the answers and sometimes it is evident when they do not happen. The problem sets will account for 25\% of the course grade. 
\subsection{Structural Economics Project}
\noindent There will be one computational project. In it, you will be instructed to solve and estimate a Structural Economics Model. Students can form groups of three to answer the problem set and hand in one answer key per group. The project will be distributed within the first two weeks of classes and will be due six weeks after. Once students hand in the project there will be a guest lecturer, Philipp Eisenhauer, who is going to talk about the project. He will talk about frontier approached to Structural Econometrics and use the project as a base example. The project will account for 25\% of the grade. Students who belong to Undergraduate or Master's programs will be welcome to hand in the project but also will be able to have their problem sets account for 50\% of the final grade.
\subsection{Final Exam}
The will be two options for the final examination.
\begin{enumerate}
\item Research Proposal + Draft
\begin{enumerate}
\item Research Proposal (due March 21st): a detailed proposal with (i) research questions; (ii) motivation and relevance; (iii) literature review; (iv) description of theoretical framework and empirical identification (including data, if necessary); (v) a time schedule for the following steps of the project. This will account for 15\% of the final grade. 
\item Paper Draft (due May 1st): a preliminary draft of the paper which includes (i) the complete basic structure of the paper; (ii) preliminary results or estimations. This will account for 35\% of the final grade.
\end{enumerate}
\item Final Oral Exam: a final oral exams that will cover all the material of the class. This will account for 50\% of the final grade. It will be schedules as permitted by the availability of Professor Heckman between March the 17th and April the 4th. 
\end{enumerate}
\subsubsection{Remarks}
\begin{enumerate}
\item You will not be able to turn in a paper draft if you do not turn in a research proposal.
\item Undergraduate and Master’s students will be welcome to hand in the Paper Draft. However, they will be able to have their research proposals account for 50\% of the final grade. 
\item You will have to notify the TAs a final exam option at the end of the fifth week of classes.
\end{enumerate}
\subsection{TAs}
\begin{itemize}
\item Jorge L. Garc\'{i}a (jorgelgarcia@uchicago.edu). Office Hours: Mondays 9:15 a.m.-10:15 a.m. in Stuart Cafeteria.  
\item John Eric Humphries (johnerichumphries@gmail.com) Office Hours: TBA. Please note that John Eric will be available as a TA beginning the sixth week of classes.
\item Yike Wang (yikewang18@uchicago.edu). Office Hours: Yike please fill this. 

\end{itemize}

































\end{document}

\begin{document}

\title{\textbf{Course Description Proposal: Econ 350, The Origins \& Consequences of Inequality in Capabilities }}
\author{Prepared by: Jorge L. Garc\'{i}a and Yike Wang \\ The University of Chicago}
\date{This Draft: \today}
\maketitle

\begin{abstract}
\noindent This course analyzes the origins and consequences of inequality in a life-cycle perspective. It uses Economic Models and Statistical Frameworks to study nine sub-topics that lead to a better understanding of a series of facts about the development of current socio-economic inequality, as described in Topic 1. In fact, the goal is to have, at the end of the course, an understanding of this phenomenon based on rigorous Economic and Econometric Methods.   
\end{abstract}

\section{General Topics}
\begin{enumerate}
\item The Dimensions of Inequality and Social Mobility
\begin{enumerate}
\item Basic Facts and Trends
\item Prices, Endowments, and Shocks
\item Skills
\item Transfers and Taxes
\item The Role of Family and Policy on Endowments
\item Philosophical \ldots
\end{enumerate}
\item Education and Inequality
\begin{enumerate}
\item Causal Inference and Instrumental Variables
\item Structural Models
\item Evidence
\item The Role of Capabilities as a Broader Notion of Skills
\end{enumerate}
\item Shocks and Uncertainty
\begin{enumerate}
\item Internal and External Variability
\item Modeling Uncertainty
\end{enumerate}
\item The Dynamics of Skill Formation
\begin{enumerate}
\item B.P. + Technology / Adult
\item The Creation of Initial Conditions 
\end{enumerate}
\item Factor Models and Models of Family Influence
\item Genes and Environments
\item Health: Life-Cycle Origins and Determinants
\item Social Mobility: Determinants and Models
\item Summary
\end{enumerate}

\section{Reading Lists}
\begin{enumerate}
\item The core reading list offers a guide to the literature in each of the class' topics and  is here: (\textbf{link to core reading list}).
\item The supplementary reading list offers students the possibility for further investigation of the class' topics and is here: (\textbf{link to supplementary reading list}).
\item The technical reading list provides background on the tools to do research on the frontier of the class' topics and is here: (\textbf{link to technical  reading list}).
\item Lecture handouts will be posted as available. Check the website and your email before class. The TAs will let you know when new handouts are available.
\item For some of the sections, there are videos from the 2012 and 2013 Summer School on Socio-Economics Inequality. These are useful as introductory summaries to the main themes of the course.
\end{enumerate}

\section{Lectures}
\noindent The class will meet in Rosenwald 301. Professor Heckman will teach on Tuesdays from 3:00 p.m. to 6:00 p.m. with a 15 minute break in between. The class will occasionally meet on Wednesdays in the same schedule. Students will need to read the relevant materials beforehand (the TAs will notify which are these materials in advance). Students will be expected to create a learning dynamics with relevant questions that stimulate discussions about the class topics. The TA session will meet in Rowsenwald 301 on Mondays from 3:00 p.m. to 6:00 p.m. The TA sessions will have two objectives: (i) cover topics not discussed in class due to time limitations; (ii) host a lecture on Structural Econometrics and Estimation by Philip Eisenhauer.

\section{Grading}
\subsection{Problem Sets}
\noindent There will be eight problem sets throughout the quarter. Students can form groups of three to answer the problem set and hand in one answer key per group. Although we cannot monitor this, we encourage students not to divide the questions and answer them individually. Group discussions improve the quality of the answers and sometimes it is evident when they do not happen. The problem sets will account for 25\% of the course grade. 
\subsection{Structural Economics Project}
\noindent There will be one computational project. In it, you will be instructed to solve and estimate a Structural Economics Model. Students can form groups of three to answer the problem set and hand in one answer key per group. The project will be distributed within the first two weeks of classes and will be due six weeks after. Once students hand in the project there will be a guest lecturer, Philipp Eisenhauer, who is going to talk about the project. He will talk about frontier approached to Structural Econometrics and use the project as a base example. The project will account for 25\% of the grade. Students who belong to Undergraduate or Master's programs will be welcome to hand in the project but also will be able to have their problem sets account for 50\% of the final grade.
\subsection{Final Exam}
The will be two options for the final examination.
\begin{enumerate}
\item Research Proposal + Draft
\begin{enumerate}
\item Research Proposal (due March 21st): a detailed proposal with (i) research questions; (ii) motivation and relevance; (iii) literature review; (iv) description of theoretical framework and empirical identification (including data, if necessary); (v) a time schedule for the following steps of the project. This will account for 15\% of the final grade. 
\item Paper Draft (due May 1st): a preliminary draft of the paper which includes (i) the complete basic structure of the paper; (ii) preliminary results or estimations. This will account for 35\% of the final grade.
\end{enumerate}
\item Final Oral Exam: a final oral exams that will cover all the material of the class. This will account for 50\% of the final grade. It will be schedules as permitted by the availability of Professor Heckman between March the 17th and April the 4th. 
\end{enumerate}
\subsubsection{Remarks}
\begin{enumerate}
\item You will not be able to turn in a paper draft if you do not turn in a research proposal.
\item Undergraduate and Master’s students will be welcome to hand in the Paper Draft. However, they will be able to have their research proposals account for 50\% of the final grade. 
\item You will have to notify the TAs a final exam option at the end of the fifth week of classes.
\end{enumerate}
\subsection{TAs}
\begin{itemize}
\item Jorge L. Garc\'{i}a (jorgelgarcia@uchicago.edu). Office Hours: Mondays 9:15 a.m.-10:15 a.m. in Stuart Cafeteria.  
\item John Eric Humphries (johnerichumphries@gmail.com) Office Hours: TBA. Please note that John Eric will be available as a TA beginning the sixth week of classes.
\item Yike Wang (yikewang18@uchicago.edu). Office Hours: Yike please fill this. 

\end{itemize}

































\end{document}

\begin{document}

\title{\textbf{Course Description Proposal: Econ 350, The Origins \& Consequences of Inequality in Capabilities }}
\author{Prepared by: Jorge L. Garc\'{i}a and Yike Wang \\ The University of Chicago}
\date{This Draft: \today}
\maketitle

\begin{abstract}
\noindent This course analyzes the origins and consequences of inequality in a life-cycle perspective. It uses Economic Models and Statistical Frameworks to study nine sub-topics that lead to a better understanding of a series of facts about the development of current socio-economic inequality, as described in Topic 1. In fact, the goal is to have, at the end of the course, an understanding of this phenomenon based on rigorous Economic and Econometric Methods.   
\end{abstract}

\section{General Topics}
\begin{enumerate}
\item The Dimensions of Inequality and Social Mobility
\begin{enumerate}
\item Basic Facts and Trends
\item Prices, Endowments, and Shocks
\item Skills
\item Transfers and Taxes
\item The Role of Family and Policy on Endowments
\item Philosophical \ldots
\end{enumerate}
\item Education and Inequality
\begin{enumerate}
\item Causal Inference and Instrumental Variables
\item Structural Models
\item Evidence
\item The Role of Capabilities as a Broader Notion of Skills
\end{enumerate}
\item Shocks and Uncertainty
\begin{enumerate}
\item Internal and External Variability
\item Modeling Uncertainty
\end{enumerate}
\item The Dynamics of Skill Formation
\begin{enumerate}
\item B.P. + Technology / Adult
\item The Creation of Initial Conditions 
\end{enumerate}
\item Factor Models and Models of Family Influence
\item Genes and Environments
\item Health: Life-Cycle Origins and Determinants
\item Social Mobility: Determinants and Models
\item Summary
\end{enumerate}

\section{Reading Lists}
\begin{enumerate}
\item The core reading list offers a guide to the literature in each of the class' topics and  is here: (\textbf{link to core reading list}).
\item The supplementary reading list offers students the possibility for further investigation of the class' topics and is here: (\textbf{link to supplementary reading list}).
\item The technical reading list provides background on the tools to do research on the frontier of the class' topics and is here: (\textbf{link to technical  reading list}).
\item Lecture handouts will be posted as available. Check the website and your email before class. The TAs will let you know when new handouts are available.
\item For some of the sections, there are videos from the 2012 and 2013 Summer School on Socio-Economics Inequality. These are useful as introductory summaries to the main themes of the course.
\end{enumerate}

\section{Lectures}
\noindent The class will meet in Rosenwald 301. Professor Heckman will teach on Tuesdays from 3:00 p.m. to 6:00 p.m. with a 15 minute break in between. The class will occasionally meet on Wednesdays in the same schedule. Students will need to read the relevant materials beforehand (the TAs will notify which are these materials in advance). Students will be expected to create a learning dynamics with relevant questions that stimulate discussions about the class topics. The TA session will meet in Rowsenwald 301 on Mondays from 3:00 p.m. to 6:00 p.m. The TA sessions will have two objectives: (i) cover topics not discussed in class due to time limitations; (ii) host a lecture on Structural Econometrics and Estimation by Philip Eisenhauer.

\section{Grading}
\subsection{Problem Sets}
\noindent There will be eight problem sets throughout the quarter. Students can form groups of three to answer the problem set and hand in one answer key per group. Although we cannot monitor this, we encourage students not to divide the questions and answer them individually. Group discussions improve the quality of the answers and sometimes it is evident when they do not happen. The problem sets will account for 25\% of the course grade. 
\subsection{Structural Economics Project}
\noindent There will be one computational project. In it, you will be instructed to solve and estimate a Structural Economics Model. Students can form groups of three to answer the problem set and hand in one answer key per group. The project will be distributed within the first two weeks of classes and will be due six weeks after. Once students hand in the project there will be a guest lecturer, Philipp Eisenhauer, who is going to talk about the project. He will talk about frontier approached to Structural Econometrics and use the project as a base example. The project will account for 25\% of the grade. Students who belong to Undergraduate or Master's programs will be welcome to hand in the project but also will be able to have their problem sets account for 50\% of the final grade.
\subsection{Final Exam}
The will be two options for the final examination.
\begin{enumerate}
\item Research Proposal + Draft
\begin{enumerate}
\item Research Proposal (due March 21st): a detailed proposal with (i) research questions; (ii) motivation and relevance; (iii) literature review; (iv) description of theoretical framework and empirical identification (including data, if necessary); (v) a time schedule for the following steps of the project. This will account for 15\% of the final grade. 
\item Paper Draft (due May 1st): a preliminary draft of the paper which includes (i) the complete basic structure of the paper; (ii) preliminary results or estimations. This will account for 35\% of the final grade.
\end{enumerate}
\item Final Oral Exam: a final oral exams that will cover all the material of the class. This will account for 50\% of the final grade. It will be schedules as permitted by the availability of Professor Heckman between March the 17th and April the 4th. 
\end{enumerate}
\subsubsection{Remarks}
\begin{enumerate}
\item You will not be able to turn in a paper draft if you do not turn in a research proposal.
\item Undergraduate and Master’s students will be welcome to hand in the Paper Draft. However, they will be able to have their research proposals account for 50\% of the final grade. 
\item You will have to notify the TAs a final exam option at the end of the fifth week of classes.
\end{enumerate}
\subsection{TAs}
\begin{itemize}
\item Jorge L. Garc\'{i}a (jorgelgarcia@uchicago.edu). Office Hours: Mondays 9:15 a.m.-10:15 a.m. in Stuart Cafeteria.  
\item John Eric Humphries (johnerichumphries@gmail.com) Office Hours: TBA. Please note that John Eric will be available as a TA beginning the sixth week of classes.
\item Yike Wang (yikewang18@uchicago.edu). Office Hours: Yike please fill this. 

\end{itemize}

































\end{document}

\begin{document}

\title{\textbf{Course Description Proposal: Econ 350,\\ Life Cycle Dynamics: The Origins \& Consequences of Inequality in Capabilities}}
\date{This Draft: \today}
\maketitle

\begin{abstract}
\noindent This course analyzes the origins and consequences of inequality in a life-cycle perspective. It uses economic models and statistical frameworks to study nine sub-topics that lead to a better understanding of a series of facts about the development of current socio-economic inequality, as described in Topic 1. In fact, the goal is to have, at the end of the course, a deeper understanding of this phenomenon based on rigorous economic and econometric methods.
\end{abstract}

\section{Topics Covered by Lectures}
\begin{enumerate}
\item The Dimensions of Inequality and Social Mobility
\begin{enumerate}
\item Basic Facts and Trends
\item Prices, Endowments, and Shocks
\item Skills
\item Transfers and Taxes
\item The Role of Family and Policy on Endowments
\item What Should Inequality Be
\item Separating Inequality from Social Mobility
\end{enumerate}
\item Education and Inequality
\begin{enumerate}
\item Causal Inference and Instrumental Variables
\item Structural Models
\item Evidence
\item The Role of Capabilities as a Broader Notion of Skills
\item The Role of Credit Constraints
\end{enumerate}
\item Shocks and Uncertainty
\begin{enumerate}
\item Internal and External Variability
\item Modeling Uncertainty
\end{enumerate}
\item The Dynamics of Skill Formation
\begin{enumerate}
\item B.P. + Technology / Adult
\item The Creation of Initial Conditions
\end{enumerate}
\item Models and Methods for the Study of Family Influence
\item Genes and Environments
\item Health: Life-Cycle Origins and Determinants
\item Social Mobility: Determinants and Models
\item Summary
\end{enumerate}

\section{Reading Lists}
\begin{enumerate}
\item The core reading list offers a guide to the literature in each of the class' topics and  is here: (\textbf{link to core reading list}).
\item The supplementary reading list offers students the possibility for further investigation of the class' topics and is here: (\textbf{link to supplementary reading list}).
\item The technical reading list provides background on the tools to do research on the frontier of the class' topics and is here: (\textbf{link to technical  reading list}).
\item Lecture handouts will be posted as available. Check the website and your email before class. The TAs will let you know when new handouts are available.
\item For some of the sections, there are videos from the 2012 and 2013 Summer School on Socio-Economics Inequality. These are useful as introductory summaries to the main themes of the course.
\end{enumerate}

\section{Lectures}
\noindent The class will meet in Rosenwald 301. Professor Heckman will teach on Tuesdays from 3:00 p.m. to 6:00 p.m. with a 15 minute break in between. The class will occasionally meet on Wednesdays in the same schedule. Students will need to read the relevant materials beforehand (the TAs will notify which are these materials in advance). Students will be expected to create a learning dynamics with relevant questions that stimulate discussions about the class topics. The TA session will meet in Rowsenwald 301 on Mondays from 3:00 p.m. to 6:00 p.m. The TA sessions will have two objectives: (i) cover topics not discussed in class due to time limitations; (ii) host a lecture on Structural Econometrics and Estimation by Philip Eisenhauer.

\section{Grading}
\subsection{Problem Sets}
\noindent There will be eight problem sets throughout the quarter. Students can form groups of three to answer the problem set and hand in one answer key per group. Although we cannot monitor this, we encourage students not to divide the questions and answer them individually. Group discussions improve the quality of the answers and sometimes it is evident when they do not happen. The problem sets will account for 25\% of the course grade.
\subsection{Structural Economics Project}
\noindent There will be one computational project. In it, you will be instructed to solve and estimate a Structural Economics Model. Students can form groups of three to answer the problem set and hand in one answer key per group. The project will be distributed within the first two weeks of classes and will be due six weeks after. Once students hand in the project there will be a guest lecturer, Philipp Eisenhauer, who is going to talk about the project. He will talk about frontier approached to Structural Econometrics and use the project as a base example. The project will account for 25\% of the grade. Students who belong to Undergraduate or Master's programs will be welcome to hand in the project but also will be able to have their problem sets account for 50\% of the final grade.
\subsection{Final Exam}
The will be two options for the final examination.
\begin{enumerate}
\item Research Proposal + Draft
\begin{enumerate}
\item Research Proposal (due March 21st): a detailed proposal with (i) research questions; (ii) motivation and relevance; (iii) literature review; (iv) description of theoretical framework and empirical identification (including data, if necessary); (v) a time schedule for the following steps of the project. This will account for 15\% of the final grade.
\item Paper Draft (due May 1st): a preliminary draft of the paper which includes (i) the complete basic structure of the paper; (ii) preliminary results or estimations. This will account for 35\% of the final grade.
\end{enumerate}
\item Final Oral Exam: a final oral exams that will cover all the material of the class. This will account for 50\% of the final grade. It will be schedules as permitted by the availability of Professor Heckman between March the 17th and April the 4th.
\end{enumerate}
\subsubsection{Remarks}
\begin{enumerate}
\item You will not be able to turn in a paper draft if you do not turn in a research proposal.
\item Undergraduate and Master’s students will be welcome to hand in the Paper Draft. However, they will be able to have their research proposals account for 50\% of the final grade.
\item You will have to notify the TAs a final exam option at the end of the fifth week of classes.
\end{enumerate}
\subsection{TAs}
\begin{itemize}
\item Jorge L. Garc\'{i}a (jorgelgarcia@uchicago.edu). Office Hours: Mondays 9:15 a.m.-10:15 a.m. in Stuart Cafeteria.
\item John Eric Humphries (johnerichumphries@gmail.com) Office Hours: TBA. Please note that John Eric will be available as a TA beginning the sixth week of classes.
\item Yike Wang (ywang18@uchicago.edu). Office Hours: Monday 12:15 p.m. - 1:15 p.m. in Ex Libris cafe (the first floor of Regenstein Library). Please email me beforehand if you plan to come.

\end{itemize}

































\end{document} 