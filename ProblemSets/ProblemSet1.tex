%Input preamble
%Style
\documentclass[11pt]{article}
\usepackage[top=1in, bottom=1in, left=1in, right=1in]{geometry}
\parindent 22pt

%Packages
\usepackage{amsmath}
\usepackage{amsfonts}
\usepackage{amssymb}
\usepackage{bm}
\usepackage{etoolbox}
\usepackage{graphicx}
\usepackage{tabularx,ragged2e,booktabs}
\usepackage{caption}
\usepackage[none]{hyphenat}
\usepackage{fixltx2e}
\usepackage[para]{threeparttable}
\usepackage[capposition=top]{floatrow}
\usepackage{subcaption}
\usepackage{pdfpages}
\usepackage{natbib}
\usepackage[colorlinks=true,linkcolor=blue,citecolor=blue,urlcolor=blue]{hyperref}
\usepackage{setspace}
\doublespacing

%Math Environments
\newtheorem{theorem}{Theorem}[section]
\newtheorem{assumption}[theorem]{Assumption}
\newtheorem{exercise}[theorem]{Exercise}
\newtheorem{example}[theorem]{Example}
\newtheorem{remark}[theorem]{Remark}
\newtheorem{definition}[theorem]{Definition}
\newenvironment{proof}[1][Proof]{\noindent\textbf{#1.} }{\ \rule{0.5em}{0.5em}}
\DeclareMathOperator{\Emax}{Emax}


\begin{document}

\title{\textbf{Problem Set 1}}
\author{Econ 350: The University of Chicago \\ Prof. James J. Heckman}
\date{This Draft: \today}
\maketitle

\section{Basic Facts on Inequality}
\subsection{The U.S. and the OECD countries}
\noindent Compare the levels and trends over time of inequality in (i) prices; (ii) hours worked; (iii) earnings; (iv) employment; (v) labor force participation; (vi) household income between the U.S. and the OECD.\\ 
\indent This is a very broad question. Try to be concrete and form a general idea of how inequality looks like and what are its major causes. (Hint: This \href{http://jenni.uchicago.edu/econ350/slides/wage-ed-facts_ECON-350_2013-01-14a_jsw.pdf}{link} and this \href{http://www.oecd-ilibrary.org.proxy.uchicago.edu/social-issues-migration-health/the-causes-of-growing-inequalities-in-oecd-countries_9789264119536-en}{other} are useful sources. Other papers in this \href{https://heckman.uchicago.edu/page/economics-350}{website} may help). 
\subsubsection{Decomposing Inequality}
Asses what is the relative importance of the following components of inequality: (i) transfer policies; (ii) male employment and hours worked; (iii) male wage rates; (iv) assortative mating; (v) female employment and hours worked; (vi) female wage rates.\\
\subsection{Education and Household Income}
\noindent How important is education as a determinant of household income?
\subsection{Income and Consumption Inequality}
Compare the difference in income and in consumption inequality. For the U.S. a good start is this \href{http://jenni.uchicago.edu/econ350/papers/Meyer_Sullivan_2012_AEIMonograph4-3.pdf}{paper}.
\section{Inequality Measures}
\noindent Define the Gini Coefficient, the variance of log income, and the Thiel-Atkinson inequality measures. For the Gini coefficient, answer the following questions: (i) does it satisfy the Dalton-Pigou principle of income transfer; (ii) is it decomposable?; (iii) is it sensitive to movements in the distribution of the relevant variable?; (iv) does it capture individual and population welfare? Hint: read the \href{http://jenni.uchicago.edu/econ350/ta_notes/archive_2010/Inequality_TA111.pdf}{handout} on Measures of Inequality.


\section{Why Should Economists Study Inequality? A Brief Survey of Distributive Justice and Ethical Duty}
\textbf{Bradley is working on a couple of questions related to his talk on philosophical aspects on inequality.}


\section{First Steps of the Empirical Project on Structural Estimation}
Solve Exercises 1.1, 1.2, and 1.3 in of the Empirical Project on Structural Estimation.
%\clearpage
%\bibliographystyle{chicago}
%\bibliography{BibtexFiles/ProblemSetsBib}
%\clearpage
\end{document}




